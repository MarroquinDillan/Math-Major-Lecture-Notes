\documentclass{article}
\usepackage[utf8]{inputenc}
\usepackage{kpfonts}
\usepackage{commath}
\usepackage{amsthm}
\usepackage{graphicx}
\usepackage[margin=0.8in]{geometry}

\newcommand{\R}{\ensuremath{\mathbb{R}}}
\newcommand{\N}{\ensuremath{\mathbb{N}}}
\newcommand{\Q}{\ensuremath{\mathbb{Q}}}
\newcommand{\B}{\ensuremath{\mathcal{B}}}
\newcommand{\wrt}{with respect to}
\newcommand{\nbd}{neighborhood}
\newcommand{\Iff}{if and only if}
\newcommand{\ts}{topological space}
\newcommand{\es}{\ensuremath{\emptyset}}
\newcommand{\coleq}{\ensuremath{\coloneqq}}
\newcommand{\powset}[1]{\ensuremath{\mathcal{P}(#1)}}
\newcommand{\define}[1]{\textbf{\underline{#1}}}
\newcommand{\card}[1]{\ensuremath{\mathbf{card} (#1)}}
\newcommand{\func}[3]{\ensuremath{#1: #2 \to #3}}
\newcommand{\closure}[1]{\ensuremath{\overline{#1}}}
\newcommand{\ball}[3]{\ensuremath{B_{#1}^{#2}(#3)}}
\newcommand{\Ball}[3]{\ensuremath{\overline{B}_{#1}^{#2}(#3)}}
\newcommand{\fRR}{\ensuremath{f:\R \to \R}}
\newcommand{\id}{\ensuremath{\mathrm{id}}}
\newcommand{\tp}{\ensuremath{\mathcal{T}}}
\newcommand{\Ts}[2]{\ensuremath{(#1,#2)}}
\newcommand{\tpcof}{\ensuremath{\tp_\text{cof}}}
\newcommand{\tpdisc}{\ensuremath{\tp_\text{disc}}}
\newcommand{\tptriv}{\ensuremath{\tp_\text{triv}}}
\newcommand{\tpeuc}{\ensuremath{\tp_\text{Euc}}}
\newcommand{\Beuc}{\ensuremath{\B_\text{Euc}}}
\newcommand{\union}{\cup}
\newcommand{\Union}{\bigcup}
\newcommand{\inter}{\cap}
\newcommand{\Inter}{\bigcap}
\newcommand{\interior}[1]{\ensuremath{\mathring{#1}}}
\newcommand{\bound}[1]{\ensuremath{\partial #1}}
\renewcommand{\Subset}{\subseteq}
\renewcommand{\Supset}{\supseteq}

\theoremstyle{definition}
\newtheorem*{defn}{Definition}
\newtheorem*{cor}{Corollary}
\newtheorem*{thm}{Theorem}
\newtheorem*{prop}{Proposition}
\newtheorem*{ex}{Ex}
\newtheorem*{lem}{Lemma}

\theoremstyle{remark}
\newtheorem*{rmk}{Remark}

\begin{document}
    \begin{center}
        \textsc{Dillan Marroquin\\MATH 440.1001\\Scribing Week 8\\Due. 22 March 2021\\}
    \end{center}
        
    \noindent\section*{\textbf{\textsc{Lecture 20}}}{
        \begin{ex}[20.2]
            Let $|X|\geq2$. Then $\Ts{X}{\tptriv}$ is not metrizable.
            \begin{proof}
                Let $x\in X$. Then $X-\{x\}\neq \es$ and $X-\{x\}\neq X$. So $X-\{x\}$ is not open which implies $\{x\}$ is not closed. Hence from P 2.1 from the midterm, $\Ts{X}{\tptriv}$ is not metrizable.
            \end{proof}
        \end{ex}
        
        \begin{ex}[20.3]
            Let $|X|=\infty$. Then $\Ts{X}{\tpcof}$ is not metrizable.
            \begin{proof}
                Note $\{x\} \Subset X$ is closed $\forall x \in X$. (This does not help determine if $\Ts{X}{\tpcof}$ is metrizable). But by PS3, suppose $U,V \Subset X$ are nonempty and open and $U \inter V=\es$. Then by definition of $\tpcof$, there exists $A,B \Subset X$ such that $|A|$ and $|B|$ are finite. So $U=X-A$ and $V=X-B$ implies $U\inter V=X-(A\inter B)=\es$. So $X$ is finite and this is a contradiction. Therefore $\Ts{X}{\tpcof}$ is not Hausdorff and thus not metrizable.
            \end{proof}
        \end{ex}
        
        \begin{ex}[20.4 \textbf{Proof Strategy}]
            Claim $\R_l=\Ts{\R}{\tp_l}$ is Hausdorff.
            \begin{proof}
                Note that $\Ts{\R}{\tpeuc}$ is Hausdorff since $\tpeuc$ comes from the metric $d(X,y)\coleq|x-y|$. Indeed, $\tpeuc<\tp_l$ be Lecture 13. Let $x\neq q \in \R$. Then there exists disjoint open subsets $U,V \in \tpeuc$ such that $x\in U$ and $y\in V$. $\tpeuc<\tp_l$ implies $U,V\in \tp_l$. Therefore, $\Ts{\R}{\tp_l}$ is Hausdorff.
            \end{proof}
            BUT $\Ts{\R}{\tp_l}$ is NOT metrizable by Lecture 14.
        \end{ex}
        
        \begin{defn}[20.5]
            Let $X$ be a \ts{}. Then
            \begin{enumerate}
                \item $X$ is \define{$T_1$} \Iff{} $\forall x\in X, \, \{x\} \Subset X$ is closed.
                \item $X$ is \define{$T_2$} \Iff{} $X$ is Hausdorff.
            \end{enumerate}
        \end{defn}
    
        \begin{prop}[20.6]\hfill
            \begin{enumerate}
                \item $T_1$ and $T_2$ properties are necessary, but NOT sufficient for $\Ts{X}{\tp}$ to be metrizable.
                \item Separation and countability properties are distinct.
            \end{enumerate}
        \end{prop}
        
        \subsection*{Converging Sequences}{
            Recall: A sequence of points $\{x_n\}_n$ in a space $X$ converges to a point $x \in X$ \Iff{} for all open subsets $U \Subset X$ such that $x \in U, \, \exists N>0$ such that $x_n\in U\ ; \forall n>N$.
            \begin{itemize}
                \item \define{Weirdest Thing Ever:} In some spaces, sequences can converge to more than 1 point!.
            \end{itemize}
            
            \begin{ex}[20.7]
                Let $X\coleq\{a,b,c\}, \, \tp\coleq \{X,\es,\{a,c\},\{b,c\},\{c\}\}$. It is easy to verify that $\tp$ is indeed a topology.\\
                Let $\{x_n\}_n$ be the sequence $x_n=c \; \forall n \geq 1$. Then $x_n \to c$ AND $x_n \to a$ ANNNDDD $x_n \to b$. Whoa.
            \end{ex}
            
            \begin{thm}[20.8]
                If $X$ is Hausdorff, then every sequence in $X$ converges to at MOST one point.
            \end{thm}
        }
    }
    \noindent\section*{\textbf{\textsc{Lecture 20}}}{
        \subsection*{More Separation Axioms}{
            \begin{defn}[20.5 Continued]
                \begin{enumerate}\hfill
                    \item $X$ is \define{$T_0$} \Iff{} for any points $x\neq y \in X$, there exist open subsets $U \Subset X$ such that $x \in U$ and $y \notin U$ OR $y \in U$ and $x \notin U$. 
                    \item $X$ is \define{$T_1$} \Iff{} $\forall x\in X, \, \{x\} \Subset X$ is closed.
                    \item $X$ is \define{$T_2$} \Iff{} $X$ is Hausdorff.
                    \item $X$ is \define{$T_3$} \Iff{} for every closed subset $A \Subset X$ and $\forall x \in X-A$, there exist open subsets $U_A,U_X \in X$ such that $A \Subset U_A$ and $x \in U_X$.
                    \item $X$ is \define{$T_4$} \Iff{} for any pair of disjoint closed subsets $A,B \Subset X$, there exist disjoint open subsets $U,V \Subset X$ such that $A \Subset U$ and $B \Subset V$.
                \end{enumerate}
            \end{defn}
            
            \begin{defn}[21.1]
                $X$ is \define{regular} \Iff{} $X$ is $T_0$ and $T_3$.\\
                $X$ is \define{normal} \Iff{} $X$ is $T_1$ and $T_4$.
            \end{defn}
            
            \begin{prop}[21.2]\hfill
                \begin{enumerate}
                    \item $T_1 \implies T_0$
                    \item $T_2 \implies T_1$
                    \item $T_1$ and $T_3 \implies T_2$
                    \item $T_1$ and $T_4 \implies T_3$
                \end{enumerate}
            \end{prop}
            
            \begin{thm}[21.3]
                If $\Ts{X}{\tp}$ is a \ts{} and metrizable, then $X$ is $T_i$ for $i=0,1,2,3,4$.
            \end{thm}
            
            \begin{proof}
                A result from Midterm Part 2 \#1 implies $X$ is $T_1$. Thm 19.1 implies that if $X$ is metrizable, then it is Hausdorff. To show $X$ is $T_3$ and $T_4$, Prop 21.2 implies that it suffices to prove $X$ is $T_4$. Let $\func{d}{X\times X}{\R}$ be a metric such that $\tp=\tp_d$ and let $A,B \Subset X$ be closed, disjoint, and nonempty subsets. $B=\closure{B} \implies \forall a \in A$, $a$ is NOT a limit point of $B$ since $a \notin B$. Therefore $\forall a \in A$ there exists $\varepsilon_a>0$ such that $\ball{a}{}{\varepsilon_a} \inter B= \es$. Similarly, $A=\closure{A} \implies \forall b \in B$ there exists $\varepsilon_b>0$ such that $\ball{b}{}{\varepsilon_b} \inter A=\es$.\\
                Define $U \coleq \Union_{a\in A} \ball{a}{}{\varepsilon_a/2}$ and $V\coleq \Union_{b\in B} \ball{b}{}{\varepsilon_b/2}$. Then $U$ and $V$ are open and $A \Subset U, \, B\Subset V$. Claim $U \inter V = \es$. Seeking a contradiction, suppose $\exists x \in U\inter V$. Then $\exists a \in A$ and $\exists b \in B$ such that $x\in \ball{a}{}{\frac{\varepsilon_a}{2}} \inter \ball{b}{}{\frac{\varepsilon_b}{2}}$. Triangle inequality implies $d(a,b)\leq d(x,a)+d(x,b)<\frac{\varepsilon_a}{2}+\frac{\varepsilon_b}{2}\leq \varepsilon_b$. Hence either $a\in \ball{b}{}{\varepsilon_b}$ if $\varepsilon_a\leq \varepsilon_b$ OR $b \in \ball{a}{}{\varepsilon_a}$ if $\varepsilon_b \leq \varepsilon_a$. This contradicts either $\ball{a}{}{\varepsilon_a} \inter B=\es$ or $\ball{b}{}{\varepsilon_b} \inter A =\es$, so $U \inter V=\es$.
            \end{proof}
        
        }
    
    }



        
\end{document}