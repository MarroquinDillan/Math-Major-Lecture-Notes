\documentclass{article}
\usepackage[utf8]{inputenc}
\usepackage{kpfonts}
\usepackage{commath}
\usepackage{amsthm}
\usepackage{graphicx}
\usepackage[margin=0.8in]{geometry}

\newcommand{\R}{\ensuremath{\mathbb{R}}}
\newcommand{\N}{\ensuremath{\mathbb{N}}}
\newcommand{\Q}{\ensuremath{\mathbb{Q}}}
\newcommand{\B}{\ensuremath{\mathcal{B}}}
\newcommand{\card}[1]{\ensuremath{\mathbf{card} (#1)}}
\newcommand{\func}[1]{\ensuremath{#1:\R \to \R}}
\newcommand{\Iff}{if and only if}
\newcommand{\nbd}{neighborhood}
\newcommand{\closure}[1]{\ensuremath{\overline{#1}}}
\newcommand{\ball}[3][\empty]{\ensuremath{B^{#1}_{#2}(#3)}}
\newcommand{\Ball}[3][\empty]{\ensuremath{\overline{B}^{#1}_{#2}(#3)}}
\newcommand{\fRR}{\ensuremath{f:\R \to \R}}
\newcommand{\define}[1]{\textbf{\underline{#1}}}
\newcommand{\ts}{topological space}
\newcommand{\tp}{\ensuremath{\mathcal{T}}}
\newcommand{\Ts}[2]{\ensuremath{(#1,#2)}}
\newcommand{\es}{\ensuremath{\emptyset}}
\newcommand{\tpcof}{\ensuremath{\tp_\text{cof}}}
\newcommand{\tpdisc}{\ensuremath{\tp_\text{disc}}}
\newcommand{\tptriv}{\ensuremath{\tp_\text{triv}}}
\newcommand{\tpeuc}{\ensuremath{\tp_\text{Euc}}}
\newcommand{\Beuc}{\ensuremath{\B_\text{Euc}}}
\newcommand{\union}{\cup}
\newcommand{\Union}{\bigcup}
\newcommand{\inter}{\cap}
\newcommand{\Inter}{\bigcap}
\newcommand{\interior}[1]{\ensuremath{\mathring{#1}}}
\newcommand{\bound}[1]{\ensuremath{\partial #1}}
\renewcommand{\Subset}{\subseteq}
\renewcommand{\Supset}{\supseteq}

\theoremstyle{definition}
\newtheorem*{defn}{Definition}
\newtheorem*{thm}{Theorem}
\newtheorem*{prop}{Proposition}
\newtheorem*{ex}{Ex}
\newtheorem*{lem}{Lemma}

\theoremstyle{remark}
\newtheorem*{rem}{Remark}

\begin{document}
    \begin{center}
        \textsc{Dillan Marroquin\\MATH 440.1001\\Scribing Week 5\\Due. 1 March 2021\\}
    \end{center}
        
    \noindent\section*{\textbf{\textsc{Lecture 12}}}
        \begin{prop}[12.1]
            Let $X$ be a set, $\B$ be a basis for a topology on $X$.
            \begin{enumerate}
                \item The collection of subsets $\tp_\B := \{U \Subset X | \forall x \in U, \exists B \in \B \text{ such that }x \in B \Subset U\}$ is a topology on $X$ generated by $\B$.
                \item A subset $U \Subset \Ts{X}{\tp_\B}$ is open \Iff{} $U$ is a union of elements in $\B$.
            \end{enumerate}
        \end{prop}
        
    I will omit the proof of Prop 12.1 which verifies $\tp_\B$ as a topology on $X$ generated by $\B$.\\
    
    \begin{prop}[12.2]
        If $\Ts{X}{\tp}$ is a \ts{} and $\B \Subset \tp$ such that $\forall U \in \tp$ and $\forall x \in U$, $\exists B \in \B$ such that $x \in B \Subset U$, then $\B$ is a basis and $\tp_\B = \tp$.
    \end{prop}

    \subsection*{Examples of $\tp_\B$}
        \begin{ex}[12.3]:
            \begin{enumerate}
                \item The Standard/Euclidean Topology on $\R$.\\
                    This has a basis $\Beuc := \{(a,b)| a<b \in \R\}$.
                
                \item The Lower Limit Topology $\R_l := \{(\R,\tp_l)\}$.\\
                    $\B_l := \{[a,b)| a<b \in \R\}$.
                
                \item The Upper Limit Topology $\R_u := \{(\R,\tp_u)\}$.\\
                    $\B_u := \{(a,b]| a<b \in \R\}$.
                    
                \item The "K-Topology" $\R_K := (\R,\tp_K)$, where $K:= \{1/n|n \in \N\}$.\\
                    $\B_K := \Beuc \union \{(a,b) - K|a<b\in \R\}$.
            \end{enumerate}
        \end{ex}
        
        \begin{lem}[Comparison Lemma 12.4]
            Let $\tp, \, \tp'$ be topologies on $X$ and let $\B, \, \B'$ be bases for $\tp, \, \tp'$ respectively. Then $\tp \Subset \tp'$ (i.e. $\tp$ is smaller) \Iff{} $\forall x \in X$ and $\forall B \in \B$ containing $x$, $\exists B' \in \B'$ such that $x \in B' \Subset B$.
        \end{lem}
        
        \begin{proof}
            ($\implies$) Let $x \in X$ and $B \in \B$ such that $x \in B$. Since every basis element is an open subset, $B \in \tp \Subset \tp'$. Therefore $B$ is an open subset of $\Ts{X}{\tp'}$. Since $\B'$ is a basis for $\tp'$, Prop. 12.1 implies that $\exists B' \in \B'$ such that $x \in B \Subset \B$.\\
            ($\impliedby$) Let $U \in \tp$. We want to show $U \in \tp'$. It suffices to show that $\forall x \in U, \, \exists B' \in \B'$ such that $x \in B' \Subset U$. Prop. 12.2 says that $U$ is the union of elements in $\B$. This implies that $\exists B \in \B$ such that $x \in B \Subset U$. By hypothesis, this implies $\exists B' \in \B'$ such that $x \in B' \Subset B \Subset U$. Thus $B' \Subset U$ as desired.
        \end{proof}
        
    \noindent\section*{\textbf{\textsc{Lecture 13}}}
        Back to examples!
        \begin{itemize}
            \item Lower Limit Topology: $\R_l := \{\R, \tp_l\}$.\\
            $\B_l := \{[a,b)|a<b\in \R\}$.
            \item Claim: $\B_l$ is a basis for a topology. We define $\tp_l$ to be generated by the basis.
        \end{itemize}
        
        \noindent Recall the axioms for a basis:
        \begin{enumerate}
            \item $\forall x \in X, \, \exists B \in \B$ such that $x \in B$.
            \item $\forall B_1,B_2 \in \B$, if $x \in B_1 \inter B_2$, then $\exists B_3 \in \B$ such that $x \in B_3 \Subset B_1 \inter B_2$.
        \end{enumerate}
        
        \begin{proof}
            (Axiom 1) Let $x \in \R$. Then $x \in [x,x+1) \in \B_l$.\\
            (Axiom 2) Let $B_1 = [a_1,b_1), \, B_2=[a_2,b_2) \in \B_l$ and suppose $x \in [a_1,b_1) \inter [a_2,b_2)$. Let $a_3 = \max\{a_1,a_2\}, \, b_3= \min\{b_1,b_2\}$. Then $x \in [a_3,b_3) \Subset [a_1,b_1) \inter [a_2,b_2)$.
        \end{proof}
        
        \begin{rem}
            $\R = \Ts{\R}{\tpeuc}, \, \Beuc := \{(a,b)|a<b\in \R\}$. Compare $\tpeuc$ to $\tp_l$ on $\R$.\\
            \underline{Claim:} $\tp_l$ is finer than $\tpeuc$, i.e. $\tpeuc \Subset \tp_l$.
        \end{rem}
        
        \begin{proof}
            Use Lem. 12.4 using $\Beuc = \B, \, \B_l = \B'$. Let $x \in \R, \, B=(a,b) \in \Beuc$. Let $x \in (a,b)$. We want to show $\exists[c,d) \in \B_l$ such that $x \in [c,d) \Subset (a,b)$. Let $c=x, \, d=b$. Then $x \in [a,b) \Subset (a,b)$.
        \end{proof}
        
        \noindent\underline{Claim:} $\tp_l \not\subseteq \tpeuc$.
        
        \begin{proof}
            We will prove this by contradiction via Lem. 12.4. Consider $B=[a,b) \in \B_l$. Let $x=a$. We want to show $\exists(c,d) \in \Beuc$ such that $a\in(c,d) \Subset [a,b)$. If $a \in (c,d)$, then $\exists \varepsilon > 0$ such that $(a-\varepsilon,a+\varepsilon) \Subset (c,d)$. But $(a-\varepsilon,a+\varepsilon) \not\subseteq [a,b)$. Therefore $(c,d) \not\subseteq [a,b)$.
        \end{proof}
        
    \noindent\section*{\textbf{\textsc{Lecture 14}}}
        Big Ideas: 
        \begin{itemize}
            \item \underline{2nd Countable Spaces}
            \item Bases for topologies are NOT unique. Sometimes you can find "small" ones.
        \end{itemize}
    
        \begin{ex}[On \R]
            $\Beuc$ is a basis for Euclidean topology on $\R$.\\
            On the other hand, consider $\widetilde{\B} := \{(q_1,q_2)| q_1<q_2 \in \Q\} \subset \Beuc$.\\
            \underline{Claim:} $\widetilde{\B}$ is another basis for $\tpeuc$ on $\R$.
        \end{ex}
    
        I will omit this proof.\\\\
        \underline{Note:} There is an injective function $\widetilde{\B} \to \Q \times \Q$ (i.e. an interval $(q_1,q_2) \mapsto (q_1,q_2)$ ordered pair.) $\implies |\widetilde{\B}| \leq |\Q\times\Q| \implies \widetilde{\B}$ is countable.
        
        \begin{defn}[14.1]
            A \ts{} $\Ts{X}{\tp}$ is \define{2nd countable} \Iff{} there exists a countable basis $\B=\{B_i\}_{i\geq 1}^\infty$ for the topology $\tp$.
        \end{defn}
        
        \begin{thm}[14.2]
            If $\Ts{X}{\tp}$ is 2nd countable, then $X$ contains a countable dense subset, i.e. $\Ts{X}{\tp}$ is separable.
        \end{thm}
        
        \begin{proof}
            Let $\B=\{B_i\}_{i=1}$ be a countable basis for $\Ts{X}{\tp}$. Then $\forall i \in \N$, choose $a_i \in B_i$. Let $D := \{a_1,a_2,\ldots\}$. Clearly $D \Subset X$ is countable by construction. Claim $D$ is dense in $X$. Let $U \Subset X$ be open and $x \in U$. Then by Prop 11.3, $\B$ is a basis which implies $\exists B_i \in \B$ such that $x \in B_i \Subset U$ which implies $a_i \in U$. This implies $D \inter U \neq \emptyset$. Therefore, $D$ is countable and dense.
        \end{proof}
        
        \begin{prop}[14.3]
            Let $\R_l$ be the Lower Limit Topology. Then
            \begin{enumerate}
                \item $\R_l$ is separable
                \item $\R_l$ is not 2nd countable
            \end{enumerate}
        \end{prop}
        
        \begin{proof}
            (1.) MATH 310 Analysis shows that $\Q$ is dense in $\R_l$.\\
            (2.) Let $\B$ be a basis for $\R_l$. We want to show $\B$ is NOT countable. We build a function $f:\R \to \B$ such that $x \mapsto B_x$. Then $x \in B_x \Subset [x,x+1) \implies |\R| \leq |\B|$. Thus $\B$ is not countable.
        \end{proof}
        
        \begin{thm}[14.4]
            Let $(X,d)$ be a metric space and let $\tp_d$ be the metric topology on $X$ induced by $d$. If $\Ts{X}{\tp_d}$ is separable, then it is 2nd countable.
        \end{thm}
\end{document}