\documentclass{article}
\usepackage[utf8]{inputenc}
\usepackage{kpfonts}
\usepackage{amsmath}
\usepackage{amsthm}
\usepackage{xfrac}
\usepackage[margin=0.6in]{geometry}

\begin{document}
    \noindent Dillan Marroquin\\ MATH 310.1002: Homework 9\\
    Due. Wed, October 28

    \begin{enumerate}
        \item Consider $f: \mathbb{R} \to \mathbb{R}$ to be a function.
        \begin{enumerate}
            \item Write what it means by definition that $\lim\limits _{x \to a^-} f(x) = \infty$.\\
            \textbf{Answer: }$\lim\limits_{x \to a^-} f(x) = \infty$ if for every $\varepsilon > 0$, there is a $\delta > 0$ such that $f(x) < \varepsilon$ whenever $x \in I \setminus \{a\}$ and $\delta < x < a$.\\
            
            \item Write what it means by definition that $\lim\limits _{x \to -\infty} f(x) = \infty$.\\
            \textbf{Answer: }$\lim\limits_{x \to -\infty} f(x) = \infty$ if for every $M > 0$, there is an $m > 0$ such that $f(x) > M$ whenever $x < m$.\\
            
            \item Write what it means by definition that $\lim\limits _{x \to \infty} f(x) = 1$.\\
            \textbf{Answer: }$\lim\limits_{x \to \infty} f(x) = 1$ if for every $\varepsilon > 0$, there is an $m > 0$ such that $|f(x)-1| < \varepsilon$ whenever $m < x$.\\
        \end{enumerate}
        
        \item Consider the function $f: \mathbb{R} \to \mathbb{R}$ defined by $f(x) =  \frac{\sqrt{x^6+3}}{x^3-1}$ for $x < 1$ and $f(x) = \frac{x-2}{x^3+\sin{\pi x}}$ for $x \geq 1$.
        \begin{enumerate}
            \item Determine the limit $\lim\limits_{x \to -\infty} f(x)$.\\
            \textbf{Answer: }
                \begin{align*}
                    \lim\limits_{x \to -\infty} \frac{\sqrt{x^6+3}}{x^3-1} = \lim\limits_{x \to -\infty} \frac{\sqrt{x^6\big(1+\frac{3}{x^6}\big)}}{x^3\big(1-\frac{1}{x^3}\big)} = \lim\limits_{x \to -\infty} \frac{\sqrt{1+\frac{3}{x^6}}}{1-\frac{1}{x^3}} = \frac{\sqrt{1}}{1} = 1.
                \end{align*}
            
            \item Determine the limit $\lim\limits_{x \to -1} f(x)$.\\
            \textbf{Answer: }\\
                \begin{align*}
                    \lim\limits_{x \to -1} \frac{\sqrt{x^6+3}}{x^3-1} = \frac{\sqrt{1+3}}{-1-1} = -1 \\
                \end{align*}
                
            \item Is $f$ continuous at $x=-1$? Justify your answer.\\
            \textbf{Answer: }Yes. $f$ is continuous at $x=-1$ because $\lim\limits_{x \to -1} f(x) = -1 = f(-1)$.\\
            
            \item Determine the limits $\lim\limits_{x \to 1^-} f(x)$ and $\lim\limits_{x \to 1^+} f(x)$.\\
            \textbf{Answer: }
                \begin{align*}
                    \lim\limits_{x \to 1^-} \frac{\sqrt{x^6+3}}{x^3-1} = \frac{\sqrt{1+3}}{0^-} = -\infty.\\
                    \lim\limits_{x \to 1^+} \frac{x-2}{x^3+\sin{\pi x}} = \frac{1-2}{1+\sin{\pi}} = -1.\\
                \end{align*}
                
            \item Is $f$ continuous at $x=1$? Justify your answer.\\
            \textbf{Answer: }No. Since $\lim\limits_{x \to 1^-}f(x) = -\infty \neq -1 = \lim\limits_{x \to 1^+} f(x)$, then $\lim \limits_{x \to 1} f(x)$ does not exist and thus is not continuous at $x = 1$.\\
            
            \item Evaluate $\lim\limits_{x \to \infty} f(x)$. Mention the theorem you are using and provide all the details.\\
            \textbf{Answer: }
                \begin{align*}
                    \lim\limits_{x \to \infty} f(x) = \lim\limits_{x \to \infty} \frac{x-2}{x^3+\sin{\pi x}}.
                \end{align*}
            Now we apply the Squeeze Theorem to the denominator of this expression. Observe that $-1 \leq \sin{\pi x} \leq 1$ implies $x^3-1 \leq x^3+\sin{\pi x} \leq x^3+1$ and that $\lim\limits x^3-1 = \lim\limits x^3+1 = \infty$. We may rewrite our initial limit:
                \begin{align*}
                    \lim\limits_{x \to \infty} \frac{x-2}{x^3+\sin{\pi x}} = \lim\limits_{x \to \infty} \frac{x-2}{x^3-1} = \lim\limits_{x \to \infty} \frac{\frac{1}{x^2}-\frac{2}{x^3}}{1-\frac{1}{x^3}} = \frac{0}{1} = 0. 
                \end{align*}
        \end{enumerate}
        
        \item Discuss the existence and the value of the limit $\lim\limits_{x \to 1} \frac{x-b}{(x-1)^2}$ for different values of the parameter $b \in \mathbb{R}$.\\
        \textbf{Answer: }We note that $\frac{x-b}{(x-1)^2}$ is continuous everywhere except when $(x-1)^2 = 0$, so we will consider values of $b$ that effect the denominator.\\
        \textbf{Case 1: }For $b = 1$, we have $\lim\limits_{x \to 1} \frac{1}{x-1}$. This limit does not exist.\\
        \textbf{Case 2: }For any $b > 1$, we have $\lim\limits_{x \to 1} \frac{x-b}{(x-1)^2}$. We now consider the left and right-sided limits to see if the limit exists.
            \begin{align*}
                \lim\limits_{x \to 1^-} \frac{x-b}{(x-1)^2} = \frac{1-b}{0^-} = \infty\\
                \lim\limits_{x \to 1^+} \frac{x-b}{(x-1)^2} = \frac{1-b}{0^+} = \infty,
            \end{align*}
        so $\lim\limits_{x \to 1} \frac{x-b}{(x-1)^2} = \infty$ for all $b < 1$.\\
        
        \textbf{Case 3: }For any $b < 1$, we have $\lim\limits_{x \to 1} \frac{x-b}{(x-1)^2}$. We now consider the left and right-sided limits to see if the limit exists.
            \begin{align*}
                \lim\limits_{x \to 1^-} \frac{x-b}{(x-1)^2} = \frac{1-b}{0^-} = -\infty\\
                \lim\limits_{x \to 1^+} \frac{x-b}{(x-1)^2} = \frac{1-b}{0^+} = -\infty,
            \end{align*}
        so $\lim\limits_{x \to 1} \frac{x-b}{(x-1)^2} = -\infty$ for all $b > 1$.\\
        
        \item Find the value(s) of the parameter $a \in \mathbb{R}$ for which the limit $\lim\limits_{x \to 1} \frac{x-a}{x^2+2x-3}$ exists.\\
        \textbf{Answer: }We first factor the denominator: $\lim\limits_{x \to 1} \frac{x-a}{x^2+2x-3} = \lim\limits_{x \to 1} \frac{x-a}{(x+3)(x-1)}$. We now observe that evaluating the limit directly at this point would cause the denominator to be $0$ and therefore not exist. If we choose $a=1$, we can avoid this: $\lim\limits_{x \to 1} \frac{x-(1)}{(x+3)(x-1)} = \lim\limits_{x \to 1} \frac{1}{x+3} = \frac{1}{4}$. So choosing $a=1$ will allow for this limit at 1 to exist.
        
        \item Evaluate the limits $\lim\limits_{x \to \infty} \frac{x^3-x^2+1}{2x^2+5}$ and $\lim\limits_{x \to -\infty} \frac{(3x+1)^2}{(2x-1)(x+2)}$.\\
        \textbf{Answer: }
        \begin{align*}
            \lim \frac{x^3-x^2+1}{2x^2+5} &= \lim \frac{x^2(x-1+\frac{1}{x^2})}{x^2(2+\frac{5}{x^2})}    &  \lim \frac{(3x+1)^2}{(2x-1)(x+2)} &= \lim \frac{9x^2+6x+1}{2x^2+3x-2}  \\
            &= \lim \frac{x-2+\frac{1}{x^2}}{2+\frac{5}{x^2}} =  \frac{\infty}{2} = \infty   &   &= \lim \frac{x^2(9+\frac{6}{x}+\frac{1}{x^2})}{x^2(2+\frac{3}{x}-\frac{2}{x^2})} = \frac{9}{2} 
        \end{align*}
        
        \item Does the limit $\lim\limits_{x \to 0} \frac{|\sin x|}{x}$ exist? Justify your answer.\\
        \textbf{Answer: }This limit does not exist. To show this, we take the one-sided limits as $x$ approaches $0$.
        \begin{align*}
            \lim_{x \to 0^-} \frac{|\sin{x}|}{x} &= -1    &     \lim_{x \to 0^+}\frac{|\sin{x}|}{x} &= \lim_{x \to 0^+} \frac{\sin{x}}{x} = 1 \\
        \end{align*}
        Since $\lim\limits_{x \to 0^-}\frac{|\sin{x}|}{x} \neq \lim\limits_{x \to 0^+}\frac{|\sin{x}|}{x}$, the limit does not exist.
    \end{enumerate}
\end{document}
