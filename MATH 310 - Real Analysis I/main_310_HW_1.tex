\documentclass{article}
\usepackage[utf8]{inputenc}
\usepackage{amsthm}
\usepackage{amsfonts}
\usepackage{amsmath}
\usepackage[margin=0.6in]{geometry}
\usepackage{amssymb}


\title{MATH 310.1002: Homework 1}
\author{Dillan Marroquin}
\date{September 2, 2020}
\begin{document}
\maketitle


    \begin{enumerate}
        \item For any $t > 0$ denote $A_t = (-t, 2 + t)$.
        
            \begin{enumerate}
                \item Determine $A_3 \setminus A_1$.
                \newline \textbf{Answer:}
                    \[A_3\setminus A_1 = (-3, 5) \setminus (-1, 3) = (-3, -1] \cup [3, 5)\]
                
                \item Determine 
                    \[\bigcap_{n=1}^{10} A_{1/n}\]
                    
                \textbf{Answer:}
                    \[\bigcap_{n=1}^{10} A_{1/n} = (-1,3) \cap (-\textstyle\frac{1}{2}, 2\frac{1}{2}) \cap (-\textstyle\frac{1}{3}, 2\frac{1}{3}) \cdots \cap (-\textstyle\frac{1}{10}, 2\frac{1}{10})\] \[= (-\textstyle\frac{1}{10}, 2\frac{1}{10})\]
                    
                \item Prove that
                 \[\bigcap_{t>0} A_t = [0,2]\]
                    \begin{proof}
                        To prove this, we must prove that each set is a subset of the other.\\
                        
                        We will begin by showing $\bigcap_{t>0} A_t \subset [0, 2]$. Let's assume that $x \in \mathbb{R}$ and that $x \notin [0, 2]$, so for example, we'll choose $x > 2$. Now let $t_0 = \frac{x-2}{2}$. We now have $x \in (-t_0, 2+t_0)$, which is a contradiction, so $\bigcap_{t>0} A_t \subset [0, 2]$. \\
                        
                        Now we will show that $[0, 2] \subset \bigcap_{t>0} A_t$. Let $y \in [0,2]$. Then $y \in (-t, 2+t)$ for all $t > 0$, so $y \in \bigcap_{t>0} A_t$.
                        
                    \end{proof}
        
            \end{enumerate}

        \item Let $f : A \to B$ and $g : B \to C$ be functions. Prove
        
            \begin{enumerate}
                \item If $g \circ f$ is onto, then $g$ is onto.
                    \begin{proof}
                        We will denote $g \circ f$ as $h$. Suppose $h$ is surjective. We wish to prove that $g$ is also surjective, or in other words, that there exists a $b \in B$ such that $g(b) = c$. By definition of $h$, there exists some $a \in A$ such that $h(a) = c$. This means that $h(f(a)) = c$. If we take $b = f(a)$, then $b \in B$ and $g(b) = c$, thus $g$ is surjective.
                        
                    \end{proof}
                    
                \item If $g \circ f$ is one-to-one, then $f$ is one-to-one.
                    \begin{proof}
                        We will denote $g \circ f$ as $h$. Suppose $h$ is injective. To prove that $f$ is also injective, we will show that for all $a, a' \in A$, $f(a) = f(a')$ implies $a = a'$. By definition of $h$, we know that for all $d, d' \in h$, $h(d) = h(d')$ implies $d = d'$. 
                    \end{proof}
            \end{enumerate}

        \item Consider the function $f : \mathbb{R} \to \mathbb{R}$, $f(x) = x^2$ and the sets $X = (-1,4)$, $Y = [1,4]$.

            \begin{enumerate}
                \item Determine $f^{-1}(X)$ and $f^{-1}(Y)$.
                \newline\textbf{Answer:}
                \[f^{-1}(X) = \varnothing\]
                \[f^{-1}(Y) = [-2, 0) \cup (0, 2]\]

                \item Determine $f(f^{-1}(X))$ and $f(f^{-1}(Y))$.
                \newline\textbf{Answer:}
                \[f(f^{-1}(X) = f(\varnothing) = \varnothing\]
                \[f(f^{-1}(Y) = f([-2, 0) \cup (0, 2]) = [1, 4]\]
            \end{enumerate}
            
        \item Prove that the function $f : D \to C$ is onto if and only if for every subset $X \subset C$ we have $f(f^{-1}(X)) = X$.
            
            \begin{proof}
                Let us first prove that if $f$ is surjective, then $f(f^{-1}(X)) = X$ for every subset $X \subset C$. 
                
                Let $y \in f(f^{-1}(X))$. Then $y = f(x)$ for some $x \in f^{-1}(X)$. So by definition we have that $x \in f^{-1}(X)$ if and only if $f(x) \in X$.
                Now let $x \in X$. Then since $f$ is surjective, $x = f(d)$ for some $d\in D$ and by definition, $d \in f^{-1}(X)$. So $d \in f(f^{-1}(X))$.\newline
                
                Now we will prove that if $f(f^{-1}(X)) = X$ for every subset $X \subset C$, then $f$ is surjective.
                
                Assume that $c \in C$ and consider the set $X = \{c\}$. Knowing that $f(f^{-1}(X)) = X$,  then there must be some $a \in f^{-1}(X)$ such that $f(a) = c$. Thus $f$ is surjective.
                
            \end{proof}
            
        \item Prove that for all $n \in \mathbb{N}$,
            \[1+3+\cdots+(2n - 1) = n^2.\]
            \begin{proof}
                We will prove by induction. First, observe that we can write $1+3+ \cdots +(2n-1)$ as $\sum_{i=1} ^n 2n-1$. For the base case, let $n = 1$. Then $1 = 1^2$ and the base case holds. Now let's assume that $\sum_{i=1} ^n 2n-1 = n^2$ for $n \in \mathbb{N}$. Then,
                    \[\Bigg(\sum_{i=1} ^n 2n - 1 \Bigg) + 2n - 1 = (n+1)^2.\]
                Substituting $n^2$ in for $\sum_{i=1} ^n 2n-1 = n^2$ gives $n^2+2n+1 = (n+1)^2$ which is of course true.
                
            \end{proof}
            
        \item Let $x_1, x_2, x_3, \ldots$ be a sequence of numbers defined recursively by
            \[x_1 = 0\text{\quad and \quad}x_{n+1} = \frac{1+x_n}{2}.\]
        Prove that $x_n < x_{n+1}$ for all $n \in \mathbb{N}$. Can you find a formula for $x_n$?
            \begin{proof}
               For our base case, let us choose $n = 1$. Then, $x_2 = \frac{1+x_1}{2} = \frac{1+0}{2} = \frac{1}{2}$. Observe that $0 < \frac{1}{2}$, so the base case holds. Now we may assume that $x_n < x_{n+1}$ for all $n \in \mathbb{N}$. We wish to prove that $x_{n+1} < x_{n+2}$ Notice that we can write $x_{n+2}$ recursively in terms of $x_{n+1}$ and ultimately achieve $x_{n+2} = \frac{1+x_{n+1}}{2}$.
                
                We can now use our previous assumption that $x_n < x_{n+1}$ to safely assume that 
                
                \[\frac{1+x_n}{2} < \frac{1+x_{n+1}}{2}\]
                \[x_{n+1} < x_{n+2}\]
                
                which thus ends our proof.
                
            \end{proof}
            
         \item \textbf{Bonus problem.} Consider the sequence defined by $a_1 = 1$ and $a_{n+1} = 2a_n + \sqrt{3a_{n}^2 - 2}$, for any $n \in \mathbb{N}$. Prove that all the terms of the sequence are positive integers.
        
            \begin{proof}
                To prove this, we will first rewrite the equation and solve for $-2$:
                \[a_{n+1} - 2a_n = \sqrt{3a_n^2 - 2}\]
                \[(a_{n+1} - 2a_n)^2 = 3a_n^2 - 2\]
                \[a_{n+1}^2 - 4a_{n+1}a_n + 4a_n^2 = 3a_n^2 - 2\]
                \[a_{n+1}^2 - 4a_{n+1}a_n + a_n^2 = -2\]
                
                (Unsure where to proceed from here) ):
            \end{proof}
            
    \end{enumerate}


\end{document}
