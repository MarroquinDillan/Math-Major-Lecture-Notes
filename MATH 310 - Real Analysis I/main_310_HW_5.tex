\documentclass{article}
\usepackage{enumitem}
\usepackage[utf8]{inputenc}
\usepackage{amsthm}
\usepackage{amsfonts}
\usepackage{amsmath}
\usepackage{paracol}
\usepackage[margin=0.7in]{geometry}
\usepackage{amssymb}

\begin{document}
    \noindent Dillan Marroquin\\
    MATH 310.1002: Homework 5\\
    30 September, 2020

    \begin{enumerate}
        \item Let $\{a_n\}_n$ be a sequence defined recursively by $a_1 = 0$ and $a_{n+1} = \sqrt{1+a_n}$ for all $n \in \mathbb{N}$.
            \begin{enumerate}
                \item Prove that $\{a_n\}_n$ is monotone.
                    \begin{proof}
                        We will prove this using induction. For the base case, let $n = 1$. Then $a_1 = 0 < \sqrt{1} = a_2$ and the base case holds true.  Assume that $a_{n+1} > a_n$ for some $n$. We want to show that $a_{n+2} > a_{n+1}$ for all $n \in \mathbb{N}$. Then
                            \[a_{n+2} = \sqrt{1+a_{n+1}} > \sqrt{1+a_n} = a_{n+1}.\]
                        So $\{a_n\}_n$ is increasing and monotone.\\
                    \end{proof}
            
                \item Prove that $\{a_n\}_n$ is bounded.
                    \begin{proof}
                        First we note that because we have already proven $\{a_n\}_n$ is increasing and begins at $a_1 = 0$, $0$ is obviously a lower bound.
                        \\Now for the upper bound, we will prove that $a_n \leq 2$ for all $n \in \mathbb{N}$ using induction. We observe that $a_1=0 \leq 2$ so the base case holds. Now assume that $a_n \leq 2$ for some $n$. We wish to prove that $a_{n+1} \leq 2$ for all $n$. Observe
                            \[a_{n+1} = \sqrt{1+a_n} \leq \sqrt{1+2} = \sqrt{3} \leq 2.\]
                    Thus, $\{a_n\}_n \leq 2$ and is bounded.\\
                    \end{proof}
                
                \item Prove that $\{a_n\}_n$ converges and find its limit.
                    \begin{proof}
                        Because we have already proven that $\{a_n\}_n$ is monotone and bounded, then $\{a_n\}_n$ is convergent by the monotone convergence theorem.\\
                        Let $l = \lim a_n$. Passing at the limit with $n \to \infty$ in $a_{n+1} = \sqrt{1+a_n}$, we get $l = \sqrt{1+l}$ and so
                            \[l^2 = 1+l\]
                            \[l^2-l-1 = 0.\]
                        So we obtain $l_1 = \frac{1+\sqrt{5}}{2}$ and $l_2 = \frac{1-\sqrt{5}}{2}$, however we know from (b) that $0 \leq a_n \leq 2$, so $0 \leq l \leq 2$. Since $l_2 < 0$, $\lim a_n = \frac{1+\sqrt{5}}{2}$.\\
                    \end{proof}
            \end{enumerate}
        \item Let $\{a_n\}_n$ be defined recursively by $a_1 = 1$ and $a_{n+1} = a_n +1/a_n$ for all $n \in \mathbb{N}$.
            \begin{enumerate}
                \item Prove that $a_n > 0$ for all $n \in \mathbb{N}$.
                    \begin{proof}
                        We will prove by induction. For our base case, choose $a_1 = 1 > 0$. Assume that $a_n > 0$ for some $n$. We wish to show that $a_{n+1} > 0$ for every $n \in \mathbb{N}$, so
                            \[a_{n+1} = a_n+\frac{1}{a_n}\]
                            \[\frac{1}{a_{n+1}} = \frac{1}{a_n}+a_n > \frac{1}{a_n}\]
                            \[a_{n+1} > a_n > 0.\]
                        Thus, $a_{n+1} > 0$ and $a_n > 0$ for all $n \in \mathbb{N}$.\\
                    \end{proof}
                
                \item Prove that the sequence is increasing.
                    \begin{proof}
                        We will prove this is increasing using induction. For the base case, let $n = 1$. Then $a_1 = 0 < \sqrt{1} = a_2$ and the base case holds true. Assume that $a_{n+1} > a_n$ for some $n$. We want to show that $a_{n+2} > a_{n+1}$ for all $n \in \mathbb{N}$. Observe that $a_{n+2} = a_{(n+1)+1}$ and that
                            \[a_{n+2} = a_{n+1}+\frac{1}{a_{n+1}} > a_n+\frac{1}{a_n} = a_{n+1}.\]
                        This proves that $a_{n+2} > a_{n+1}$.\\
                    \end{proof}
                
                \item Prove that the sequence is unbounded.
                    \begin{proof}
                        We will prove by contradiction. Assume that $\{a_n\}_n$ is bounded. Then by the monotone convergence theorem, $a_n$ converges to $L \in \mathbb{R}$. Passing at the limit with $n \to \infty$ in $a_{n+1} = a_n+\frac{1}{a_n}$, we get $L=L+\frac{1}{L}$. So $\frac{1}{L} = 0$ and $L = \infty$. This is a contradiction, so $\{a_n\}_n$ is unbounded.\\
                    \end{proof}
                
            \end{enumerate}

        \item  Let $\{a_n\}_n$ be defined recursively by $a_1 = 7$ and $a_{n+1} = 2-1/a_n$ for all $n \in \mathbb{N}$.
            \begin{enumerate}
                \item Prove that $\{a_n\}_n$ is monotone.
                    \begin{proof}
                        We will prove using induction. Observe that $a_1 = 7 > a_2 = \frac{13}{7}$, so the base case holds. We wish to prove that $a_{n+1} \geq a_{n+2}$ for all $n \in \mathbb{N}$. Assume that $a_n \geq a_{n+1}$ for some $n$. Then, $\frac{1}{a_n} \geq \frac{1}{a_{n+1}}$. So, $a_{n+1} = 2-\frac{1}{a_n} \geq 2-\frac{1}{a_{n+1}} = a_{n+2}$ and thus $\{a_n\}_n$ is monotone.\\
                    \end{proof}
                
                \item Prove that $\{a_n\}_n$ is bounded.
                    \begin{proof}
                        We will prove that $1 \leq a_n \leq 7$ by induction. Assume that $a_n \in [1,7]$ for a certain $n \in \mathbb{N}$. Then, 
                            \[a_{n+1} = 2 - \frac{1}{a_n} \geq 2-\frac{1}{1} = 1 \text{ and,}\]
                            \[a_{n+1} = 2-\frac{1}{a_n} \leq 2-\frac{1}{7} = \frac{13}{7} < 7.\]
                        Thus, $\{a_n\}_n$ is bounded.\\
                    \end{proof}
                
                \item Prove that $\{a_n\}_n$ converges and find its limit.
                    \begin{proof}
                    By the monotone convergence theorem, $\{a_n\}_n$ is convergent.\\
                    Let $l = \lim_{n \to \infty} a_n \in \mathbb{R}$. Then,
                        \[\lim_{n \to \infty} a_{n+1} = 2-\frac{1}{a_n}\]
                        \[l = 2-\frac{1}{l}\]
                        \[l^2-2l+1 = 0.\]
                    Solving this equation for $l$, we achieve $l = 1$, so $\lim_{n \to \infty} a_n = 1$.\\
                    \end{proof}
            \end{enumerate}
        
        \item Show using the definition that the sequence $\{\ln{(n^2+1)}\}_n$ diverges to $\infty$.\\
        Definition: If $a_n$ is a sequence of real numbers, then we say $\lim a_n = \infty$ if for every real number $M$ there is a number $N$ such that $a_n > M$ whenever $n > N$.
            \begin{proof}
                To prove this sequence diverges, we must choose an $N\in \mathbb{R}$ such that $\ln{(n^2+1)} > M \in \mathbb{R}$ whenever $n > N$.\\
                Choose $N = \sqrt{2e^M-1}$. Then $\ln{(n^2+1)} > \ln{\big(\sqrt{2^M-1}^2+1\big)} = \ln{(2e^m)} = \ln{2}+M > M$.\\
            \end{proof}
        
        \item Prove that $2^n \geq n^2$ for all $n \in \mathbb{N}$, $n \geq 4$. Determine
            \[\lim_{n \to \infty} \frac{2^n}{n\sqrt{n}}\]
            \begin{proof}
                We will first prove $2^n \geq n^2$ for all $n \in \mathbb{N}$, $n\geq 4$ using induction. For the base case choose $n = 4$. Then $2^4 = 16 \geq n^2$ and the base case is true. Assume $2^n \geq n^2$ for some $n \geq 4$. We want to show that $2^{n+1} \geq (n+1)^2$ for all $n \geq 4$. Observe that
                    \[2^{n+1}=2^n \cdot 2 \geq 2n^2 \geq n^2+2n+1 = (n+1)^2,\]
                thus $2^n \geq n^2$ for all $n \geq 4$.
            \end{proof}
            \textbf{Answer: }
                \[\frac{2^n}{n\sqrt{n}} \geq \frac{n^2}{n\sqrt{n}} = \frac{(n^2)(n\sqrt{n})}{(n\sqrt{n})(n\sqrt{n)}} = \frac{n^3\sqrt{n}}{n^3} = \sqrt{n}.\]
            But $\lim \sqrt{n} = \infty$, so $\lim \frac{2^n}{n\sqrt{n}} > \infty$; thus $\lim\frac{2^n}{n\sqrt{n}} = \infty$.
    \end{enumerate}
\end{document}
