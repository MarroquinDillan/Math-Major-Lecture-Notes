\documentclass{article}
\usepackage{enumitem}
\usepackage[utf8]{inputenc}
\usepackage{amsthm}
\usepackage{amsfonts}
\usepackage{amsmath}
\usepackage{paracol}
\usepackage[margin=0.6in]{geometry}
\usepackage{amssymb}

\begin{document}
    \noindent Dillan Marroquin\\ MATH 310.1002: Homework 6\\
    Due. Wed, October 7

    \begin{enumerate}
        \item Let $\{a_n\}_n$ be a sequence. Suppose that there exists $M > 0$ and $r \in (0,1)$ such that $|a_n - a_{n+1}| < Mr^n$ for all $n \in \mathbb{N}$. Prove that $\{a_n\}_n$ converges. (\textit{Hint: }First prove that $\{a_n\}_n$ is a Cauchy sequence.)\\
        \textbf{Answer: }We wish to prove that $\{a_n\}_n$ is Cauchy, i.e. that for any $\varepsilon > 0$, there is a $N_\varepsilon$ such that $|a_n-a_m| \leq \varepsilon$ for all $n,m \geq N_\varepsilon$. It is given that $|a_n-a_{n+1}| < Mr^n$, so then 
            \begin{align*}
                n &\to n+1: |a_{n+1}-a_{n+2}| < Mr^{n+1}\\
                n &\to n+2: |a_{n+2}-a_{n+3}| < Mr^{n+2}\\
                \vdots\\
                n &\to n+k-1: |a_{n+k-1}-a_{n+k}| < Mr^{n+k-1},
            \end{align*}
        for all $n \in \mathbb{N}$. Adding the above inequalities gives
            \[|a_n-a_{n+1}|+|a_{n+1}-a_{n+2}|+|a_{n+2}-a_{n+3}|+ \cdots +|a_{n+k-1}-a_{n+k}| < Mr^n + \cdots Mr^{n+k-1}.\]
        By the triangle inequality, we get the following:
            \begin{align*}
                |a_n-a_{n+k}| &= |(a_n-a_{n+1})+(a_{n+1}-a_{n+2})+(a_{n+2}-a_{n+3})+ \cdots +(a_{n+k-1}-a_{n+k})|\\
                 &\leq |a_n-a_{n+1}|+|a_{n+1}-a_{n+2}|+|a_{n+2}-a_{n+3}|+ \cdots +|a_{n+k-1}-a_{n+k}|\\
                 &< Mr^n(1+r+r^2+\cdots+r^{k-1}) = Mr^n\frac{1-r^k}{1-r} < \frac{M}{1-r}r^n.
            \end{align*}
        Solving the previous inequality for $n$ gives
            \[r^n < \frac{\varepsilon(1-r)}{M}\]
            \[n\ln{r} < \ln{\frac{\varepsilon(1-r)}{M}}\]
            \[n > \frac{\ln{\frac{\varepsilon(1-r)}{M}}}{\ln{r}}.\]
        Let $N_\varepsilon = \left \lfloor{\dfrac{\ln{\frac{\varepsilon(1-r)}{M}}}{\ln{r}}} \right \rfloor + 1$. Then for every $n,m \geq N_\varepsilon$, we have $|a_n-a_m| < \varepsilon$.\\
        
        \item Let $\{a_n\}_n$ be a sequence such that $|a_n - a_{n+1}| \to 0$. Must $\{a_n\}_n$ converge? If so, prove it, and if not, find a counterexample.\\
        \textbf{Answer: }No, $\{a_n\}_n$ does not have to converge.
            \begin{proof}
                To prove that this sequence does not have to converge, we will provide a counterexample. Let $a_{n+1}-a_n = \frac{1}{n+1}$ for all $n \in \mathbb{N}$, which does converge to $0$. Then we have the following:
                    \begin{align*}
                        a_n-a_{n-1} &= \frac{1}{n}\\
                        a_{n-1}-a_{n-2} &= \frac{1}{n-1}\\
                        a_{n-2}-a_{n-3} &= \frac{1}{n-2}\\
                        \vdots\\
                        a_2-a_1 &= \frac{1}{2}. 
                    \end{align*}
                Adding together the above equations gives $a_n-a_1 = \frac{1}{2}+\frac{1}{3}+\frac{1}{4}+\frac{1}{5}+\cdots+\frac{1}{n-2}+\frac{1}{n-1}+\frac{1}{n}$. Solving for $a_n$ then gives $a_n = a_1+ \frac{1}{2}+\frac{1}{3}+\frac{1}{4}+\frac{1}{5}+\cdots+\frac{1}{n-2}+\frac{1}{n-1}+\frac{1}{n}$. Observe that this is a harmonic series that diverges to infinity, so $\{a_n\}_n$ does not converge.\\
            \end{proof}
        
        \item Prove that the sequence $\{a_n\}_n$ given below is not a Cauchy sequence.
        \[a_n = (-1)^n \frac{n+1}{3n}.\]
        
        \item Find three convergent subsequences of $\{a_n\}_n$ with distinct limits and find these limits. Is $\{a_n\}_n$ Cauchy?
        \[a_n = \frac{1}{n} + \cos{\frac{n\pi}{3}}.\]
        \textbf{Answer: }
            \begin{align*}
                a_{6n} &= \frac{1}{6n}+\cos{\frac{6n\pi}{3}} = \frac{1}{6n}+\cos{(2n\pi)} = \frac{1}{6n}+1 \to 1\\
                a_{6n+1} &= \frac{1}{6n+1}+\cos{\frac{(6n+1)\pi}{3}} = \frac{1}{6n+1}+\cos{\Big(2n\pi+\frac{\pi}{3}\Big)} = \frac{1}{6n+1}+ \frac{1}{2} \to \frac{1}{2}\\
                a_{6n+2} &= \frac{1}{6n+2}+\cos{\frac{(6n+2)\pi}{3}} = \frac{1}{6n+2}+\cos{\Big(\frac{2\pi}{3}\Big)} = \frac{1}{6n+2}-\frac{1}{2} \to -\frac{1}{2}
            \end{align*}
        Because the subsequences of $\{a_n\}_n$ all approach different values as $n \to \infty$, $\{a_n\}_n$ is not Cauchy.\\
        
        \item In each case find $\limsup_n a_n$ and $\liminf_n{a_n}$.
            \begin{enumerate}
            \item $a_n = 1+(-1)^n \frac{2n+3}{n}$\\
            \textbf{Answer: }We will consider the 2 subsequences $a_{2n}$ and $a_{2n+1}$.
                \begin{align*}
                    a_{2n} &= 1+(-1)^{2n}\Big(\frac{2(2n)+3}{2n}\Big) = 1+1\Big(\frac{4n+3}{2n}\Big) \implies \lim \Big[1+1\Big(\frac{4n+3}{2n}\Big)\Big] = 1+2 = 3\\
                    a_{2n+1} &= 1+(-1)^{2n+1}\Big(\frac{2(2n+1)+3}{2n+1}\Big) = 1-1\Big(\frac{4n+5}{2n}\Big) \implies \lim \Big[1-1\Big(\frac{4n+5}{2n}\Big)\Big] = 1-2 = -1.
                \end{align*}
            So, $\limsup_n{a_n} = \lim a_{2n} = 3$ and $\liminf_n{a_n} = \lim a_{2n+1} = -1$.\\
            
            \item $a_n = \cos{\frac{n\pi}{3}}$\\
            \textbf{Answer: }
                \begin{align*}
                    a_1 &= \cos{\frac{\pi}{3}} = \frac{1}{2},\; a_2 = \cos{\frac{2\pi}{3}} = -\frac{1}{2},\; a_3 = \cos{\pi} = -1,\; a_4 = -\frac{1}{2},\; a_5 = \frac{1}{2}, \; a_6 = 1\\
                    \liminf{x_n} &= \sup_n \inf{x_k} = \sup_n \inf\Big\{-1,-\frac{1}{2}, \frac{1}{2},1\Big\} = \sup_n\{-1\} = -1\\
                    \limsup{x_n} &= \inf_n \sup{x_k} = \inf_n \sup\Big\{-1,-\frac{1}{2}, \frac{1}{2},1\Big\} = \inf_n\{1\} = 1
                \end{align*}
            
            \item $a_n = \frac{((-1)^n-2)^n}{2^n}$\\
            \textbf{Answer: }We will consider the 2 subsequences $a_{2n}$ and $a_{2n+1}$.
                \begin{align*}
                    a_{2n} &= \frac{[(-1)^{2n}-2]^{2n}}{2^{2n}} = \frac{(1-2)^{2n}}{2^{2n}} = \frac{(-1)^{2n}}{2^{2n}} = \frac{1}{2^{2n}}\\
                    a_{2n+1} &= \frac{[(-1)^{2n+1}-2]^{2n+1}}{2^{2n+1}} = \frac{(-1-2)^{2n+1}}{2^{2n+1}} = \frac{(-3)^{2n+1}}{2^{2n+1}} = -\Big(\frac{3}{2}\Big)^{2n+1}.
                \end{align*}
            $\lim a_{2n} = \lim \frac{1}{2^{2n}} = 0$ and $\lim a_{2n+1} = \lim \Big[-\Big(\frac{3}{2}\Big)^{2n+1}\Big] = -\infty$. So,\\
            $\limsup_n{a_n} = \lim a_{2n} = 0$ and $\liminf_n{a_n} = \lim a_{2n+1} = -\infty$.
            \end{enumerate}
 
    \end{enumerate}
\end{document}
