\documentclass{article}
\usepackage[utf8]{inputenc}
\usepackage{kpfonts}
\usepackage[mathscr]{euscript}
\usepackage{commath}
\usepackage{bbm}
\usepackage{enumerate}
\usepackage{amsthm}
\usepackage{graphicx}
\usepackage{wasysym}
\usepackage[margin=0.8in]{geometry}

\newcommand{\N}{\mathbbm{N}}
\newcommand{\Z}{\mathbbm{Z}}
\newcommand{\Q}{\mathbbm{Q}}
\newcommand{\R}{\mathbbm{R}}
\newcommand{\C}{\mathbbm{C}}
\newcommand{\es}{\emptyset}
\newcommand{\union}{\cup}
\newcommand{\Union}{\bigcup}
\newcommand{\inter}{\cap}
\newcommand{\Inter}{\bigcap}
\newcommand{\coleq}{\coloneqq}
\newcommand{\script}[1]{\mathscr{#1}}
\newcommand{\powset}[1]{\mathcal{P}(#1)}
\newcommand{\id}{\mathrm{id}}
\newcommand{\inverse}[1]{#1^{-1}}
\newcommand{\define}[1]{\textbf{\underline{#1}}}
\newcommand{\func}[3]{#1: #2 \to #3}
\newcommand{\lcm}{\mathrm{lcm}}
\renewcommand{\mod}[1]{\ (\mathrm{mod}\ #1)}
\renewcommand{\Subset}{\subseteq}
\renewcommand{\Supset}{\supseteq}
\renewcommand{\qedsymbol}{$\blacksquare$}

\theoremstyle{definition}
\newtheorem*{defn}{Definition}
\newtheorem*{cor}{Corollary}
\newtheorem*{thm}{Theorem}
\newtheorem*{prop}{Proposition}
\newtheorem*{ex}{Example}
\newtheorem*{lem}{Lemma}
\theoremstyle{remark}
\newtheorem*{rmk}{Remark}

%Abstract Algebra specific commands
\newcommand{\Znx}{(\mathbb{Z}/n)^\times}
\newcommand{\Rx}{\mathbb{R}^\times}
\newcommand{\cyc}[1]{\langle#1\rangle}
\newcommand{\im}{\mathrm{im}}
\newcommand{\normal}{\unlhd}
\newcommand{\ideal}{\unlhd}
\newcommand{\iso}{\cong}
\newcommand{\sgn}{\mathrm{sgn}}
\newcommand{\ev}{\mathrm{ev}}
\newcommand{\K}{\mathbbm{K}}
\renewcommand{\H}{\mathbbm{H}}

\begin{document}
    \begin{center}
        \textsc{Dillan Marroquin\\MATH 331.1001\\Scribing Week 14\\Due. 29 November 2021\\}
    \end{center}
        
    \section*{Lecture 35}{
        An ideal $I \ideal R$ is \define{principal} iff $\exists a \in I$such that $I=(a)=\{ra|r\in R\}$.
        
        \begin{ex}
            \begin{enumerate}
                \item Let $R=\Z$, $I=n\Z=(n)$.
                \item For every ring, the zero ideal is principal and $R$ is a principal ideal (i.e. $\{0_R\}=(0)$), $R=(1)$.)
            \end{enumerate}
        \end{ex}
        
        \begin{defn}[35.1]
            A ring $R$ is a \define{principal ideal ring (PIR)} iff every ideal of $R$ is principal.\\
            $R$ is a \define{principal ideal domain (PID)} iff $R$ is an integral domain and $R$ is a PIR.
        \end{defn}
        
        Recall Theorem 34.4 which stated that $\Z$ is a PID (wasn't worded like this).
        
        \begin{prop}[35.2]
            $\forall n>1$, $\Z/n/Z$ is a PIR.
        \end{prop}
        
        \begin{prop}[35.3]
            A field $\K$ has exactly 2 ideals: the zero ideal and $\K$.
        \end{prop}
        
        \begin{cor}[35.4]
            \begin{enumerate}
                \item A field is a PID.
                \item If $\K$ is a field and $\func{\varphi}{\K}{S}$ is a ring homomorphism, then $\varphi$ is injective OR $S$ is the zero ring.
                \item $\Z/n\Z$ is a PID if $n$ is prime.
            \end{enumerate}
        \end{cor}
        
        \begin{rmk}[35.5]
            \begin{enumerate}
                \item Nice Theorem in Paulin: "If $R$ is a finite integral domain, then $R$ is a field." So, $\Z/n\Z$ being an integral domain implies $(\Z/n\Z)^\times=\{[k]|\gcd(k,n)=1\}=\{[1],[2],\ldots,[n-1]\}$ since this is a field. Therefore if $d|n$ and $d<n$, then $d=1$. Thus $n$ is prime and we conclude $\Z/n\Z$ is a PID iff $\Z/n\Z$ is a field iff $n=p$ prime.
                \item $\Z[x]$ is an integral domain by Theorem 33.4, but is not a PID.
            \end{enumerate}
        \end{rmk}
        
        \subsection*{How to Get More Examples of PIDs?}{
            Recall in Theorem 13.3 (wayyyy back then), we showed that every subgroup of a cyclic group is cyclic. Crucial in our proof was the division algorithm in $\Z$.
            
            \begin{defn}[35.6]
                Let $R$ be a commutative ring such that $0\neq 1$.
                \begin{enumerate}
                    \item A \define{Euclidean function} on $R$ is a set-theoretic function $\func{N}{R-\{0_R\}}{\N\union \{0\}}$ such that
                    \begin{enumerate}
                        \item $\forall a\in R$, $\forall b\in R-\{0_R\}$, $N(a)\leq N(ab)$.
                        \item $\forall a \in R$ and $\forall b\neq 0\in R$, $\exists q,r\in R$ such that $a=bq+r$ with either $r=0$ OR $N(r)<N(b)$.
                    \end{enumerate}
                    \item An integral domain is the \define{Euclidean domain} iff $R$ admits a Euclidean function.
                \end{enumerate}
            \end{defn}
            
            \begin{thm}[35.7]
                The following are Euclidean domains:
                \begin{enumerate}
                    \item $\Z$ with $N(m)\coleq |m| \ \forall m\neq 0$ (absolute value).
                    \item Any field $\K$ with $N(a)\coleq 1 \ \forall a\neq 0\in \K$.
                    \item $\Z[i]$ with $N(k+ib)\coleq a^2+b^2 \ \forall a+ib\neq 0$.
                    \item Polynomial Ring $\K[x]$ with coefficients in a field $\K$, $N(f)\coleq \deg(f) \ \forall f\neq 0$.
                \end{enumerate}
            \end{thm}
        
        }
    }
\end{document} 