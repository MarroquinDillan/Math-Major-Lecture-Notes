\documentclass{article}
\usepackage[utf8]{inputenc}
\usepackage{kpfonts}
\usepackage[mathscr]{euscript}
\usepackage{commath}
\usepackage{bbm}
\usepackage{enumerate}
\usepackage{amsthm}
\usepackage{graphicx}
\usepackage{wasysym}
\usepackage[margin=0.8in]{geometry}

\newcommand{\N}{\mathbbm{N}}
\newcommand{\Z}{\mathbbm{Z}}
\newcommand{\Q}{\mathbbm{Q}}
\newcommand{\R}{\mathbbm{R}}
\newcommand{\C}{\mathbbm{C}}
\newcommand{\es}{\emptyset}
\newcommand{\union}{\cup}
\newcommand{\Union}{\bigcup}
\newcommand{\inter}{\cap}
\newcommand{\Inter}{\bigcap}
\newcommand{\coleq}{\coloneqq}
\newcommand{\script}[1]{\mathscr{#1}}
\newcommand{\powset}[1]{\mathcal{P}(#1)}
\newcommand{\id}{\mathrm{id}}
\newcommand{\inverse}[1]{#1^{-1}}
\newcommand{\define}[1]{\textbf{\underline{#1}}}
\newcommand{\func}[3]{#1: #2 \to #3}
\newcommand{\lcm}{\mathrm{lcm}}
\renewcommand{\mod}[1]{\ (\mathrm{mod}\ #1)}
\renewcommand{\Subset}{\subseteq}
\renewcommand{\Supset}{\supseteq}
\renewcommand{\qedsymbol}{$\blacksquare$}

\theoremstyle{definition}
\newtheorem*{defn}{Definition}
\newtheorem*{cor}{Corollary}
\newtheorem*{thm}{Theorem}
\newtheorem*{prop}{Proposition}
\newtheorem*{ex}{Example}
\newtheorem*{lem}{Lemma}
\theoremstyle{remark}
\newtheorem*{rmk}{Remark}

%Abstract Algebra specific commands
\newcommand{\Znx}{(\mathbb{Z}/n)^\times}
\newcommand{\Rx}{\mathbb{R}^\times}
\newcommand{\cyc}[1]{\langle#1\rangle}
\newcommand{\im}{\mathrm{im}}
\newcommand{\normal}{\unlhd}
\newcommand{\ideal}{\unlhd}
\newcommand{\iso}{\cong}
\newcommand{\sgn}{\mathrm{sgn}}
\newcommand{\ev}{\mathrm{ev}}
\newcommand{\K}{\mathbbm{K}}
\renewcommand{\H}{\mathbbm{H}}

\begin{document}
    \begin{center}
        \textsc{Dillan Marroquin\\MATH 331.1001\\Scribing Week 15\\Due. 7 December 2021\\}
    \end{center}
        
    \section*{Lecture 36}{
        \begin{prop}[36.1]
            Every ED is a PID.
        \end{prop}
        
        \begin{ex}
            \begin{defn}[36.2]
                A $\deg(n)$ polynomial is \define{monic} iff its leading coefficient is $1_R$ i.e. $f=x^n+a_{n-1}x^{n-1}+\cdots+a_0$.
            \end{defn}
            \begin{cor}[36.3]
                If $\K$ is a field and $I \ideal \K[x]$ is a non-trivial ideal, then $\exists f\neq 0 \in I$ such that $I=(f)$, $f$ is monic, and $\deg(f) \leq \deg(g) \ \forall g\in I-\{0\}$.
            \end{cor}
        \end{ex}
        
        \subsection*{Primes and Irreducibles}{
            \begin{defn}[36.4]
                \begin{enumerate}
                    \item An element $a \in R$ \define{divides} $b\in R$ iff $\exists r\ in R$ such that $b=ra$. Write $a|b$.
                    \item An element $p \in R$ is \define{prime} iff
                    \begin{enumerate}
                        \item $p\neq 0$ and $p$ is not a unit (i.e. $p$ has no multiplicative inverse).
                        \item Whenever $p|ab$, then $p|a$ or $p|b$.
                    \end{enumerate}
                    \item An element $c \in R$ is \define{irreducible} iff
                    \begin{enumerate}
                        \item $c\neq 0$ and $c$ is not a unit.
                        \item If $c=ab$, then either $a$ or $b$ is a unit.
                    \end{enumerate}
                \end{enumerate}
            \end{defn}
            
            \begin{prop}[36.5]
                If $R$ is an integral domain, then prime is irreducible.
            \end{prop}
        }
    }
    \section*{Lecture 37}{
        \begin{ex}
            Claim: Let $q=\pm \in \Z$ with $p$ a prime number. Then
            \begin{itemize}
                \item $q$ is a prime element of $\Z$ and
                \item $q$ is an irreducible element of $\
                Z$.
            \end{itemize}
        \end{ex}
        
        \begin{ex}
            Recall that a non-constant polynomial $f \in \K[x]$ is irreducible if it can't be factored into $2$ non-constant polynomials, i.e. if $f=gh$, then $\deg(g)=0$ or $\deg(h)=0$.\\
            \define{Fact:} Units of $\K[x]=$ non-zero constants $=\K^\times$. Hence $f$ an irreducible polynomial implies $f$ is an irreducible element in $\K[x]$.
            \begin{ex}
                $\forall a\in \K$, $x-a$ is irreducible in $\K[x]$. Note that $x^2+1$ is irreducible in $\R[x]$, but $x^2+1$ is NOT irreducible in $\K[x]$ if $\sqrt{-1}\in \K$.
            \end{ex}
        \end{ex}
        
        \begin{ex}
            Let $R=\Z[\sqrt{-5}]=\{a+b\sqrt{-5}|a,b \in \Z\}$. Note that $a+b\sqrt{-5}$ is a unit in $R$ iff $a^2+5b^2=1$. Also, $3=3+0\sqrt{-5}\in R$ is irreducible. BUT $3$ is not prime!! Let $a=2+\sqrt{-5}$, $b=2-\sqrt{-5} \in R$. Then $ab=4+5=9$. Therefore, $3|ab$, but in $\Z[\sqrt{-5}]$, $3\nmid 2+\sqrt{-5}$ and $3\nmid 2-\sqrt{-5}$. Therefore, irreducible element does not imply prime element.
        \end{ex}
        
        \begin{ex}
            Claim: Let $n\geq 2 \in \N$ be composite, $p$ be a prime number such that $p|n$. Then $[p]\in \Z/n\Z$ is a prime element.
        \end{ex}
    }
    \section*{Lecture 38}{
        \begin{rmk}
            Recall in $\Z$, $q= \pm p$ with $p$ a prime number. Then in $q$ is prime and irreducible $\Z^\times =\{\pm 1\}$. Slogan: "Primeness doesn't care about multiplying by units."
        \end{rmk}
        
        \begin{defn}[38.1]
            Let $R$ be a commutative ring not equal to $0$.
            \begin{enumerate}
                \item An ideal $P \ideal R$ is \define{prime ideal} iff for $P\neq R$ whenever, $ab \in P$, then either $a \in P$ or $b \in P$.
                \item An ideal $M \ideal R$ is \define{maximal} iff $M\neq R$ and whenever $J \ideal R$ is an ideal with $M \Subset J$, then either $J=R$ or $J=M$.
            \end{enumerate}
        \end{defn}
        
    \begin{ex}[38.2]
        $\K$ is a field, $f \in \K[x]$ irreducible polynomial (e.g. $f=x-1$). Claim: $(f)\ideal \K[x]$ is maximal.
        \begin{enumerate}
            \item Note $1 \notin (f)$ since $\deg1=0$ and $\deg f\geq 1$.
            \item Suppose $J\ideal \K[x]$ such that $(f)\Subset J$. Proposition 36.1 implies $\exists g \in J$ such that $J=(g)$. $(f)\Subset J$ implies $f \in (g)$. Therefore $\exists h \in \K[x]$ such that $f=gh$. $f$ irreducible implies either $g$ or $h$ is a non-zero constant.
            \begin{enumerate}
                \item If $g$ is a unit, then $J=(g)=(1)=\K[x]$.
                \item If $h$ is a unit, then $\exists \inverse{h} \in \K^\times$ such that $\inverse{h}h=1$ which implies $g=\inverse{h}f$. Therefore $J=(g)=(\inverse{h}f)=(f)$.
            \end{enumerate}
        \end{enumerate}
    \end{ex}
    
    \begin{thm}[38.3]
        Let $R$ be a commutative non-trivial ring, and $I \ideal R$ be an ideal. Then
        \begin{enumerate}
            \item $I$ is a prime ideal iff $R/I$ is an integral domain.
            \item $I$ is maximal ideal iff $R/I$ is a field.
        \end{enumerate}
    \end{thm}
    
    \begin{rmk}[38.4]
        Since every field is an integral domain, Theorem 38.3 implies every maximal ideal is a prime ideal.
    \end{rmk}
    
    \begin{thm}[38.5]
        Let $R$ be a PID. Then $p \in R$ is prime iff $p$ is irreducible.
    \end{thm}
    
    }
\end{document} 