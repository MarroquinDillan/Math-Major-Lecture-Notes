\documentclass{article}
\usepackage[utf8]{inputenc}
\usepackage{kpfonts}
\usepackage[mathscr]{euscript}
\usepackage{commath}
\usepackage{bbm}
\usepackage{enumerate}
\usepackage{amsthm}
\usepackage{graphicx}
\usepackage{wasysym}
\usepackage[margin=0.8in]{geometry}

\newcommand{\N}{\mathbbm{N}}
\newcommand{\Z}{\mathbbm{Z}}
\newcommand{\Q}{\mathbbm{Q}}
\newcommand{\R}{\mathbbm{R}}
\newcommand{\C}{\mathbbm{C}}
\newcommand{\es}{\emptyset}
\newcommand{\union}{\cup}
\newcommand{\Union}{\bigcup}
\newcommand{\inter}{\cap}
\newcommand{\Inter}{\bigcap}
\newcommand{\coleq}{\coloneqq}
\newcommand{\script}[1]{\mathscr{#1}}
\newcommand{\powset}[1]{\mathcal{P}(#1)}
\newcommand{\id}{\mathrm{id}}
\newcommand{\inverse}[1]{#1^{-1}}
\newcommand{\define}[1]{\textbf{\underline{#1}}}
\newcommand{\func}[3]{#1: #2 \to #3}
\newcommand{\lcm}{\mathrm{lcm}}
\renewcommand{\mod}[1]{\ (\mathrm{mod}\ #1)}
\renewcommand{\Subset}{\subseteq}
\renewcommand{\Supset}{\supseteq}
\renewcommand{\qedsymbol}{$\blacksquare$}

\theoremstyle{definition}
\newtheorem*{defn}{Definition}
\newtheorem*{cor}{Corollary}
\newtheorem*{thm}{Theorem}
\newtheorem*{prop}{Proposition}
\newtheorem*{ex}{Example}
\newtheorem*{lem}{Lemma}
\theoremstyle{remark}
\newtheorem*{rmk}{Remark}

%Abstract Algebra specific commands
\newcommand{\Znx}{(\mathbb{Z}/n)^\times}
\newcommand{\Rx}{\mathbb{R}^\times}
\newcommand{\cyc}[1]{\langle#1\rangle}
\newcommand{\im}{\mathrm{im}}
\newcommand{\normal}{\unlhd}
\newcommand{\iso}{\cong}
\newcommand{\sgn}{\mathrm{sgn}}
\newcommand{\ev}{\mathrm{ev}}

\begin{document}
    \begin{center}
        \textsc{Dillan Marroquin\\MATH 331.1001\\Scribing Week 11\\Due. 8 November 2021\\}
    \end{center}
        
    \section*{\textbf{\textsc{Lecture 26}}}{
        \subsection*{Paulin Chapter 4: Rings!}{
            \noindent Idea: Study objects like $(\Z,+,0,*,1)$, develop an abstract notion of primes and the fundamental theorem of arithmetic.
            
            \begin{defn}[26.1]
                A \define{ring} $\mathbf{(R,+,0,*,1)}$ is a set $R$ equipped with binary operators $\func{+,*}{R\times R}{R}$ and elements $0,1\in R$ such that
                \begin{enumerate}
                    \item $(R,+,0)$ is an abelian group,
                    \item $(R,*,1)$ is a monoid (i.e. a group where multiplicative inverses may not exist),
                    \item Left/Right distributive law holds: $\forall a,b,c\in R$, $(a+b)*c=a*c+b*c$ and $a*(b+c)=a*b+a*c$.
                \end{enumerate}
            \end{defn}
            
            \noindent \define{Notation:} $ab\coleq a*b$ and $\forall n\geq 0 \in Z$, $na\coleq a+a\cdots+a$ ($n$ times) and $a^n\coleq a*a*\cdots *a$ ($n$ times). Note that $na\neq a^n$ in general.
            
            \begin{defn}[26.2]
                A ring $R$ is commutative iff $\forall a,b \in R$, $a*b=b*a$.
            \end{defn}
        }
        \subsection*{Basic Examples of Rings}{
            \begin{enumerate}
                \item $\Z,\Q,\R,\C$. Commutative.
                \item $(\Z/n,\overline{+},\overline{0},\overline{*},\overline{1})$. Commutative.
                \item The Zero Ring $R=\{0_R\}$, where $1_R=0_R$. Commutative.
                \item $M_n(\R)\coleq \{n\times n\text{ matrices with entries in }\R\}$, $(M_n(\R),+,0_n,*,I_n)$. Non-commutative for $n\geq 2$.
                \item $\script{C}([0,1])\coleq \{\func{f}{[0,1]}{\R}|f\text{ is continuous}\}$. In this ring, $(f+g)(x)\coleq f(x)+g(x)$, $(fg)(x)\coleq f(x)g(x)$, $0(x)\coleq 0 \in \R$, $1(x) \coleq 1 \in \R$ $\forall x \in [0,1]$.
            \end{enumerate}
        }
        \subsection*{Abstract Properties of Rings}{
            \begin{prop}[26.3]
                Let $R$ be a ring.
                \begin{enumerate}
                    \item $\forall n,m \geq 1$, let $a_1,\ldots,a_n \in R$ and $b_1,\ldots, b_m \in R$. Then $\left(\sum_{i=1}^n a_i\right)\cdot\left(\sum_{j=1}^m b_j\right)=\sum_{i=1}^n\sum_{j=1}^m a_i b_j$.
                    \item $\forall a\in R$, $a*0=0=0*a$.
                    \item $\forall a,b\in R$, $a(-b)=-a(b)=-ab$, where $-b,-a$ are the additive inverses of $b,a$ respectively. In particular, $(-a)(-b)=ab$.
                \end{enumerate}
            \end{prop}
        }
        \subsection*{Important Example: Polynomial Rings}{
            Let $R$ be a commutative ring. Then
            \[R[x]\coleq\{a_0+a_1x+a_2x^2+\cdots+a_{n-1}x^{n-1}+a_nx^n|\forall n\geq 0 \ a_i \in R\}=\text{"}R\text{ adjoin }x\text{"}.\]
        Let $f,g\in R[x]$. Write $f=\sum_{i=0}^n a_i x^i$, $g=\sum_{j=0}^m b_j x^j$. WLOG, assume $m\leq n$. Define $b_{m+1}=b_{m+2}=\cdots=b_n=0 \in R$, then $f+g\coleq \sum_{i=0}^n (a_i+b_i)x^i$. Also, $fg \coleq \sum_{k=0}^{m+n} c_k x^k$, where $c_k\coleq \sum_{l=0}^k a_lb_{k-l}$. 
        }
    }
    \section*{Lecture 27}{
        \define{Additive Identity:} $0\coleq \sum_i a_ix^i$, $a_i=o \in R$ $\forall i \geq 0$.\\
     
        \noindent\define{Multiplicative Identity:} $1\coleq \sum_i a_ix^i$, $a_o=1 \in R$, $a_i=0 \in R$ $\forall i \geq 1$.
     
        \begin{prop}[27.1]
            $R$ commutative implies $R[x]$ is commutative.
        \end{prop}
     
        \begin{rmk}
            $R[x][y]$. This is just a polynomial in $2$ variables.
        \end{rmk}
        
        \begin{defn}[27.2]
            Let $f=\sum a_kx^k \in R[x]$, where $\sum a_k x^k$. Then the \define{degree} of $f$, $\deg(f) \in \N$ is the largest $n \in \Z$ such that $a_n\neq 0$. Often, $\deg(0)\coleq -\infty$.
        \end{defn}
        
        \subsection*{Basic Constructions}{
            \begin{defn}[27.3]
                Let $R$ be a ring. A subset $S \Subset R$ is a \define{subring} iff
                \begin{enumerate}
                    \item $(S,+,0_R) \leq (R,+,0_R)$ is a subgroup with respect to $+$.
                    \item $\forall x,y \in S$, $x*y \in S$. i.e. $S$ is closed under multiplication.
                    \item $1_R \in S$.
                \end{enumerate}
                We write $S\leq R$ to denote that $S$ is a subring of $R$.
            \end{defn}
        }
            \begin{ex}
                \begin{enumerate}
                    \item We have $\Z \leq \Q \leq \R \leq \C$.
                    \item Let $R$ be commutative. Then $R \leq R[x]$.
                    \item (Non-Commutative Examples): Let $R=M_2(\R)$ and $S=\left\{A\in R|A=\alpha=\big(\begin{smallmatrix} a_1&a_2 \\ 0&3 \end{smallmatrix}\big)\right\}$. Then $S \leq R$.
                \end{enumerate}
            \end{ex}
            
            \noindent\define{CAUTION!!!} Some authors\ldots
            \begin{enumerate}
                \item don't require a ring to have $1$ (multiplicative identity)
                \item don't require subrings to have $1_R \in S$ (no Axiom 3).
            \end{enumerate}
        
        \subsection*{Basic Constructions}{
            \begin{enumerate}
                \item $n\Z \not\leq \Z$, $n>1$ since $1 \notin \Z$.
                \item If $R \neq \{0_R\}$, then $\{0_R\}\not\leq R$ since $1_R \notin \{0_R\}$.
                \item Take $S=\{f=\sum a_ix^2 \in R[x]|a_0=0\}\not\leq R[x]$ since $1 \notin R[x]$.
            \end{enumerate}
        }
    }
    \section*{Lecture 28}{
        \subsection*{Ring Homomorphisms}{
            \begin{defn}[28.1]
                Let $R$, $S$ be rings. A \define{ring homomorphism} from $R$ to $S$ is a function $\func{\varphi}{R}{S}$ such that $\forall a$,$b \in R$,
                \begin{enumerate}
                    \item $\varphi(a+b)=\varphi(a)+\varphi(b)$,
                    \item $\varphi(ab)=\varphi(a)\varphi(b)$, and
                    \item $\varphi(1_R)=1_S$.
                A \define{ring isomorphism} is a ring homomorphism $\varphi$ such that $\varphi$ is a bijection.
                \end{enumerate}
            \end{defn}
            
            \begin{ex}
                \begin{enumerate}
                    \item $\func{\id}{R}{R}$ is a ring isomorphism. BOOOORING!!!
                    \item Let $n>1$. Then $\func{\pi}{\Z}{\Z/n}$, $\pi(a)\coleq[a]$ is a ring homomorphism.
                    \item (NON-EXAMPLE) Let $\func{\det}{M_2(\R)}{\R}$ be a function. Then Axioms 2 and 3 are satisfied, but not Axiom 1 since $\det(A+B)\neq \det(A)+\det(B)$ in general.
                \end{enumerate}
            \end{ex}
            
            \begin{prop}[28.2]
                Let $r \in R$. The function $\ev_r(f)\coleq f(r)$ is a ring homomorphism ("evaluation at $r$").
            \end{prop}
            
            In general, elements of $R[x]$ "aren't functions."
            
            \begin{ex}
                $\Z/2[x]$.
                \begin{align*}
                    \deg(-\infty)&: \overline{0}   &    \deg(1)&: x,x+\overline{1}\\
                    \deg(0)&: \overline{1}   &   \deg(2)&: x^2,x^2+x,x^2+\overline{1},x^2+x+\overline{1}.
                \end{align*}
                The number of $\ev$ homomorphisms is $2$: $\func{\ev_{\overline{0}},\ev_{\overline{1}}}{\Z/2[x]}{\Z/2}$.\\
                
                \noindent Let $f\coleq x^2+x+\overline{1}$, $g\coleq \overline{1}$. Then $\ev_{\overline{0}}(f)=\overline{1}$, $\ev_{\overline{1}}(f)=\overline{1}^2+\overline{1}+\overline{1}=\overline{1}$.\\
                Also, $\ev_{\overline{0}}(g)=\overline{1}$, $\ev_{\overline{1}}(g)=\overline{1}$, BUT $f\neq g$.
            \end{ex}
            
            \begin{defn}[28.3]
                Let $\func{\varphi}{R}{S}$ be a ring homomorphism. The \define{kernel} of $\varphi$ is the subset $\ker(\varphi)\coleq\{r\in R|\varphi(r)=0_S\}$ of $R$.\\
                The \define{image} of $\varphi$ is the subset $\im(\varphi)\coleq\{\varphi(r)|r \in R\}$ of $S$.
            \end{defn}
            
            \begin{prop}[28.4]
                \begin{enumerate}
                    \item $\im(\varphi)\leq S$ is a subgroup of $S$.
                    \item $\ker(\varphi)\leq R$ is a subring of $R$ iff $S=\{0_S\}$ is the trivial ring.
                \end{enumerate}
            \end{prop}
            
        }
    }
\end{document} 