\documentclass{article}
\usepackage[utf8]{inputenc}
\usepackage{kpfonts}
\usepackage[mathscr]{euscript}
\usepackage{commath}
\usepackage{enumerate}
\usepackage{amsthm}
\usepackage{graphicx}
\usepackage[margin=0.8in]{geometry}

\newcommand{\N}{\mathbb{N}}
\newcommand{\Z}{\mathbb{Z}}
\newcommand{\Q}{\mathbb{Q}}
\newcommand{\R}{\mathbb{R}}
\newcommand{\C}{\mathbb{C}}
\newcommand{\es}{\emptyset}
\newcommand{\union}{\cup}
\newcommand{\Union}{\bigcup}
\newcommand{\inter}{\cap}
\newcommand{\Inter}{\bigcap}
\newcommand{\coleq}{\coloneqq}
\newcommand{\script}[1]{\mathscr{#1}}
\newcommand{\powset}[1]{\mathcal{P}(#1)}
\newcommand{\id}{\mathrm{id}}
\newcommand{\inverse}[1]{#1^{-1}}
\newcommand{\define}[1]{\textbf{\underline{#1}}}
\newcommand{\func}[3]{#1: #2 \to #3}
\renewcommand{\mod}[1]{\ (\mathrm{mod}\ #1)}
\renewcommand{\Subset}{\subseteq}
\renewcommand{\Supset}{\supseteq}
\renewcommand{\qedsymbol}{$\blacksquare$}

%Abstract Algebra specific commands
\newcommand{\Znx}{(\mathbb{Z}/n)^\times}
\newcommand{\gen}[1]{\langle#1\rangle}
\newcommand{\im}[1]{\mathrm{im}#1}

\theoremstyle{definition}
\newtheorem*{defn}{Definition}
\newtheorem*{cor}{Corollary}
\newtheorem*{thm}{Theorem}
\newtheorem*{prop}{Proposition}
\newtheorem*{ex}{Example}
\newtheorem*{lem}{Lemma}

\theoremstyle{remark}
\newtheorem*{rmk}{Remark}

\begin{document}
    \begin{center}
        \textsc{Dillan Marroquin\\MATH 331.1001\\Midterm Review}
    \end{center}
        
    \noindent\section*{\textbf{\textsc{Definitions}}}{
        \begin{defn}
            A \define{group} $(G,*,e)$ is a set $G$ equipped with a binary operator $*$ and an identity element $e \in G$ such that the following hold:
            \begin{enumerate}
                \item Associativity: $(ab)c=a(bc) \, \forall a,b,c \in G$
                \item Existence of Identity: $\exists e\in G$ such that $ae=ea=a \forall a \in G$
                \item Existence of Inverses: Given $a \in G, , \exists b \in G$ such that $ab=ba=e$.
            \end{enumerate}
        \end{defn}
        
        \begin{defn}
            Let $(G,*)$ be a group. A \define{subgroup} of $G$ is a subset $H \subset G$ such that
            \begin{enumerate}
                \item $e\in H$
                \item $x,y \in H \Rightarrow x*y \in H$
                \item $x\in H \Rightarrow \inverse{x} \in H$.
            \end{enumerate}
        \end{defn}
        
        \begin{defn}
            Let $G$ be a group and $a \in G$. The subset $\gen{a}\coleq \{a^k| k \in \Z\}$ is a subgroup of $G$ called the \define{cyclic subgroup} generated by $a$. A group is \define{cyclic} iff $a \in G$ such that $G=\gen{a}$.
        \end{defn}
        
        \begin{defn}
            Let $(G,*,e_G)$ and $(H,\circ,e_H)$ be groups. A \define{group homomorphism} between $G$ and $H$ is a function $\func{\varphi}{G}{H}$ such that $\forall a,b, \in G, \, \varphi(a*b)=\varphi(a)\circ\varphi(b)$.
        \end{defn}
        
        \begin{defn}
            A function $\func{\varphi}{G}{H}$ is a \define{group isomorphism} iff $\varphi$ is bijective and a homomorphism. $G\cong H$ iff $G$ and $H$ are isomorphic.
        \end{defn}
        
        \begin{defn}
            For $x \in G$, the \define{left coset} containing $x$ is $xH\coleq\{xh|h\in H\} \subset G$. (Note that $y \in xH$ implies $yH=xH$). 
        \end{defn}
        
        \begin{defn}
            Let $G$ be a group, $H\leq G$ a subgroup. Denote $G/H$ as the \define{SET of left cosets} of $H$ in $G$. The size of this set is the \define{index} of $H$ in $G$, denoted $[G:H]=|G/H|$.
        \end{defn}
        
        \begin{defn}
            Let $\func{\varphi}{G}{H}$ be a group homomorphism. The \define{kernel} of $\varphi$ is the subset of $G$ $\ker\varphi=\{x \in G|\varphi(x)=e_H\}$.
        \end{defn}
    }
    \noindent\section*{\textbf{\textsc{Examples}}}{
        \begin{ex}
            $\Z/n\coleq \{[0],[1],\ldots,[n-1]\}$ set of equivalence classes of $\equiv \mod n$ on $\Z$. 
        \end{ex}
        
        \begin{ex}
            $GL_n(\R)\coleq\{\text{$n\times n$ matrix $A$}|\det(A)\neq 0\}$. $(GL_n(\R),\cdot,I_n)$ is an abelian group.
        \end{ex}
    }
    \noindent\section*{\textbf{\textsc{Theorems}}}{
        \begin{prop}[8.3 ($\Leftarrow$ Direction)]
            Let $G$ be a group, and $H\subseteq G$ a subset. Then $H$ is a subgroup iff $H \neq \es$ and $\forall a,b \in H$, $a\inverse{b} \in H$.
        \end{prop}
        
        \begin{proof}
            $(\Leftarrow)$ Assume $H\neq \es$ and $\forall a,b \in H, \, a\inverse{b}\in H$. Observe that $H \neq \es$ implies $\exists x \in H$. Let $a=b=x$. Then $x\inverse{x}=e \in H$. Now verify Axiom 3: Let $x\in H$. We want to show $\inverse{x}\in H$. Let $a=e$ and $b=x$. Then $a\inverse{b}=e\inverse{x}\in H$ by assumption. This implies $\inverse{x}\in H$. For Axiom 2, let $x,y\in H$. Set $a=x$. We know $\inverse{y}\in H$ by proof of Axiom 3. Therefore $x(\inverse{(\inverse{y})}=xy \in H$.
        \end{proof}
        
        \begin{thm}[13.1]
            If $|G|=p$ for $p$ prime, then $G$ is cyclic. In particular, $\forall a \in G-\{e\}$, $G=\gen{a}$.
        \end{thm}
        
        \begin{proof}
            Let $a\in G\setminus\{e\}$. Corollary 12.6 (if $G$ is a finite group, then $\forall a \in G, \, |a|\big||G|$) implies $|a|\big|p$ since $p=|G|$. Therefore $|a|=1$ or $|a|=p$. Since $a\neq e$, then $|a|=p$. Proposition 12.5 (if $|a|=n$ for $a \in G$, then $|\gen{a}|=|a|$) implies $|\gen{a}|=p=|G|$. Therefore $G=\gen{a}$. 
        \end{proof}
        
        \begin{thm}[15.3]
            If $G=\gen{a}$ is cyclic order $n$, then $G\cong \Z/n$.
        \end{thm}
        
        \begin{proof}
            Suffices to construct a group isomorphism $\func{\varphi}{\Z/n}{G}$. Let $\varphi([k])=a^k$. First we check if $\varphi$ is well-defined.\\
            Suppose $l\in [k]$. WTS $\varphi([k])=\varphi([l])$, i.e. that $a^k=a^l$. We have $l \in [k]$ which implies $l\equiv k\mod n$. Therefore $\exists m \in \Z$ such that $l-k=nm$ which implies $a^{l-k}=a^{nm}=(a^n)^m=e^m=e$. Therefore $a^{l-k}=e$ which implies $a^l=a^k$.\\
            Now we show $\varphi$ is a group homomorphism: $\varphi([k]+[l])=\varphi([k+l])=a^{k+l}=a^ka^l=\varphi([k])+\varphi([l])$.\\
            Next we show $\varphi$ is surjective. Note that the image of $\varphi(\Z/n)=\{\varphi([k]|[k] \in \Z/n\}=\{a^k|[k] \in \Z/n\}=G$. Thus $\varphi$ is surjective.\\
            Finally, we show $\varphi$ is injective: Suppose $\varphi([k])=\varphi([l])$. Then $a^k=a^l$ which implies $a^{k-l}=e$ and by Lemma 14.3 (), we have $n|k-l$. So $k\equiv l\mod n$ and thus $[k]=[l]$.
        \end{proof}
    }
    \noindent\section*{\textbf{\textsc{Problem Sets}}}{
    
    }
\end{document} 