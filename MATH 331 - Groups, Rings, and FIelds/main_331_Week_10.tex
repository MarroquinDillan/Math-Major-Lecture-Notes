\documentclass{article}
\usepackage[utf8]{inputenc}
\usepackage{kpfonts}
\usepackage[mathscr]{euscript}
\usepackage{commath}
\usepackage{bbm}
\usepackage{enumerate}
\usepackage{amsthm}
\usepackage{graphicx}
\usepackage{wasysym}
\usepackage[margin=0.8in]{geometry}

\newcommand{\N}{\mathbbm{N}}
\newcommand{\Z}{\mathbbm{Z}}
\newcommand{\Q}{\mathbbm{Q}}
\newcommand{\R}{\mathbbm{R}}
\newcommand{\C}{\mathbbm{C}}
\newcommand{\es}{\emptyset}
\newcommand{\union}{\cup}
\newcommand{\Union}{\bigcup}
\newcommand{\inter}{\cap}
\newcommand{\Inter}{\bigcap}
\newcommand{\coleq}{\coloneqq}
\newcommand{\script}[1]{\mathscr{#1}}
\newcommand{\powset}[1]{\mathcal{P}(#1)}
\newcommand{\id}{\mathrm{id}}
\newcommand{\inverse}[1]{#1^{-1}}
\newcommand{\define}[1]{\textbf{\underline{#1}}}
\newcommand{\func}[3]{#1: #2 \to #3}
\newcommand{\lcm}{\mathrm{lcm}}
\renewcommand{\mod}[1]{\ (\mathrm{mod}\ #1)}
\renewcommand{\Subset}{\subseteq}
\renewcommand{\Supset}{\supseteq}
\renewcommand{\qedsymbol}{$\blacksquare$}

\theoremstyle{definition}
\newtheorem*{defn}{Definition}
\newtheorem*{cor}{Corollary}
\newtheorem*{thm}{Theorem}
\newtheorem*{prop}{Proposition}
\newtheorem*{ex}{Example}
\newtheorem*{lem}{Lemma}
\theoremstyle{remark}
\newtheorem*{rmk}{Remark}

%Abstract Algebra specific commands
\newcommand{\Znx}{(\mathbb{Z}/n)^\times}
\newcommand{\Rx}{\mathbb{R}^\times}
\newcommand{\cyc}[1]{\langle#1\rangle}
\newcommand{\im}{\mathrm{im}}
\newcommand{\normal}{\unlhd}
\newcommand{\iso}{\cong}
\newcommand{\sgn}{\mathrm{sgn}}

\begin{document}
    \begin{center}
        \textsc{Dillan Marroquin\\MATH 331.1001\\Scribing Week 10\\Due. 1 November 2021\\}
    \end{center}
        
    \section*{\textbf{\textsc{Lecture 24}}}{
        We will begin with a proof of Corollary 23.3 from last lecture\ldots
        \begin{proof}[Proof Cor 23.3]
            By Proposition 23.2, all $\sigma$ commute with one another. Thus $\alpha^l=\sigma_1^l\sigma_2^l\cdots\sigma_k^l$ $\forall l\geq 1$. As $\{\sigma_i\}$ is disjoint for all $i$, $\inverse{\sigma_i}$ and $\sigma_{j\neq1}^l$ are disjoint also. Then $\alpha^m=e$ iff $\sigma_i^m=e$ $\forall i=1,\ldots,k$. Lemma 14.2 implies $|\sigma_i|\big|m$. Thus $\alpha^m=e$ iff $m$ is a multiple of $|\sigma_1|,|\sigma_2|,\ldots,|\sigma_k|$. By definition of order, $|\alpha|$ must be the smallest such that $\alpha^m=e$, so $|\alpha^m|=e$, so $|\alpha|=\lcm\{|\sigma_i|\}_{i=1}^k$.
        \end{proof}
        
        \subsection*{Generators of $S_n$}{
            Here, we are looking for the set of elements of $\{\sigma_1,\ldots,\sigma_n\}\Subset S_n$ such that every element of $S_n$ can be written as $\sigma_1^{k_1},\ldots,\sigma_n^{k_n}$ for $k_1,\ldots,k_n \in \Z$.
        
            \begin{prop}[24.1]
                Let $n\geq 2$ and $\sigma=(i_1\ldots i_k)\in S_n$ be a $k$-cycle. Then $\sigma$ can be written as a product of transpositions. In particular, $\sigma=(i_1\;i_k)(i_1\;i_{k-1})\ldots(i_1\;i_2)$ (i.e. $k-1$ transpositions).
            \end{prop}
            
            \begin{rmk}[Ex 24.2]
                Consider $\sigma=(1\;2\;3\;4)\in S_4$. Then $(1\;4)(1\;3)(1\;2)=(1\;2\;3\;4)$. Note that $1\mapsto2$, $2\mapsto1\mapsto3$, and $3\mapsto1\mapsto4$.\\
                \textbf{Note:} Decompositions into transpositions are note unique! For example, $(1\;2)(2\;3)(1\;2)(3\;4)(1\;2)=(1\;2\;3\;4)$ as well.
            \end{rmk}
            
            \begin{thm}[24.3]
                Every non-identity element of $S_n$ is uniquely (up to rearrangement) a product of disjoint cycles, each of length 2.
            \end{thm}
            
            \noindent This is how we define our cycle notation: $\alpha=\big(\begin{smallmatrix} 1&2&3 \ \ 4&5&6 \\ 3&2&1 \ \ 5&6&4\end{smallmatrix}\big) \in S_6=(1\;3)(4\;5\;6)$.
            
            \begin{cor}[24.4]
                For all $n\geq2$, $S_n$ is generated by the set of transpositions $\{(ij)\in S_n|1\leq i<j\leq n\}$.
            \end{cor}
            
            \begin{rmk}
                We also know that $S_3$ is generated by $\sigma=(1\;2\;3)$ and $\tau=(2\;3)$. We can also show that $S_{n>3}$ is generated by $\sigma=(1\;2\;3\;\ldots\;n-1\;n)(n-1\;n)$.
            \end{rmk}
        }
    }
    \section*{\textbf{\textsc{Lecture 25}}}{
        \subsection*{Sign of Permutation}{
            \begin{defn}[25.1]
                Let $\sigma\in S_n$. We say $\sigma$ is \define{even/odd} iff $\sigma$ can be written as an even/odd number of transpositions. We write $\sgn(\sigma)\coleq +1$ if $\sigma$ is even or $-1$ if $\sigma$ is odd.
            \end{defn}
            
            \begin{ex}
                If $\sigma=(i_1\;i_2\;\ldots\;i_k)\in S_n$ is a $k$-cycle, then $\sigma$ is even if $k$ is odd, or odd if $k$ is even.\\
                Proposition 24.1 implies $\sigma=(i_1\;i_k)(i_1\;i_{k-1})\cdots(i_1\;i_2)$.
            \end{ex}
            
            \begin{thm}[25.2]
                A permutation can't be both odd and even. In particular, $\func{\sgn}{S_n}{\{\pm1\}}$ is well-defined.
            \end{thm}
            
            \noindent Evidence for Theorem 25.2: Let $\Vec{e_1},\ldots,\Vec{e_n}$ be a standard basis of $\R^n$. So $\Vec{e_1}=[1\;0\;0\;\cdots\;0],\Vec{e_2}=[0\;1\;0\;\cdots\;0],\ldots$. To each $\sigma\in S_n \mapsto n\times n$ matrix $P_\sigma$:\\
            $P_\sigma\coleq[\Vec{e_{\sigma(1)}},\Vec{e_{\sigma(2)}},\ldots,\Vec{e_{\sigma(n)}}]$.
            
            \begin{ex}
                \begin{enumerate}
                    \item $S_2=\{(1),(1\;2)\}$. We have $(1)\mapsto \big(\begin{smallmatrix} 1&0 \\ 0&1\end{smallmatrix}\big)$, $(1\;2)\mapsto \big(\begin{smallmatrix} 0&1 \\ 1&0\end{smallmatrix}\big)$.
                    \item $\sigma=(1\;2\;3)\in S_3\mapsto P_\sigma=\Big(\begin{smallmatrix} 0&0&1 \\ 1&0&0 \\ 0&1&0\end{smallmatrix}\Big)$. Note that $\det(P_\sigma)=1=\sgn(\sigma)$ since $\sigma$ is a 3-cycle.
                \end{enumerate}    
            \end{ex}
            
            \noindent \define{Fact:} $S_n \to \mathrm{GL}_n(\R)$ is a group homomorphism.\\
            
            \noindent If an $n\times n$ matrix $A=[\Vec{a_1}\;\Vec{a_2}\ldots\;\Vec{a_n}]$, then swapping any 2-columns changes the sign of the determinate.\\
        
            \noindent\define{Fact:} $\sgn(\sigma)=\det(P_\sigma)$.
        }
    }
\end{document} 