\documentclass{article}
\usepackage{amsmath}
\usepackage{commath}
\usepackage{units}
\usepackage{kpfonts}
\usepackage[margin=0.6in]{geometry}
\usepackage{multicol}

\begin{document}
    \noindent STAT 461.1002: Homework 6\\
    Dillan Marroquin\\
    27 October, 2020\\

    \noindent \textbf{Section 3.6 Problems}
    \begin{enumerate}
        \item 3.6.5. Use Theorem 3.6.1 (Let $W$ be any random variable, discrete or continuous, having mean $\mu$ and for which $E(W^2)$ is finite. Then $\mathrm{Var}(W) = \sigma^2 = E(W^2)-\mu^2$) to find the variance of the random variable $Y$, where $f_Y(y) = 3(1-y)^2, \; 0 < y < 1$.\\
        \textbf{Answer: }
        \begin{align*}
            \mu &= \int_0^1 y \cdot f_Y(y) \dif y = \int_0^1 3y(1-y)^2 \dif y = 3\int_0^1 y^3-2y^2+y \dif y = 3\bigg(\frac{y^4}{4}-\frac{2y^3}{3}+\frac{y^2}{2}\bigg)\bigg|_0^1 = \frac{1}{4}.\\
            E[Y^2] &= \int_0^1 y^2 \cdot f_Y(y) \dif y = \int_0^1 3y^2(1-y)^2 \dif y = 3\int_0^1 y^4-2y^3+y^2 \dif y = 3\bigg(\frac{y^5}{5}-\frac{y^4}{2}+\frac{y^3}{3}\bigg)\bigg|_0^1 = \frac{1}{10}.
        \end{align*}
        So, $\mathrm{Var}(Y) = E[Y^2]-\mu^2 = \frac{1}{10}-\frac{1}{16} = \frac{3}{80}$.\\
        
        \item \textbf{(461 only)} 3.6.11. Suppose that $Y$ is an exponential random variable, so $f_Y(y) = \lambda e^{-\lambda y}, \ y \geq 0$. Show that the variance of $Y$ is $1/\lambda^2$.\\
        \textbf{Answer: }\\
        We first find $\mu$ using integration by parts:
        \begin{align*}
            \mu &= \int_0^\infty y\lambda e^{-\lambda y} \dif y = -ye^{-\lambda y}\Big|_0^\infty - \int_0^\infty -\frac{e^{-\lambda y}}{\lambda} \dif y.
        \end{align*}
        We now use L'Hospital's Rule to evaluate $\lim\limits_{y \to \infty} -ye^{-\lambda y} = \lim \frac{-y}{e^{\lambda y}} = \lim \frac{-1}{\lambda e^{\lambda y}} = \frac{-1}{\infty} = 0$. So,
        \begin{align*}
           \mu = \int_0^\infty \frac{e^{-\lambda y}}{\lambda} \dif y = \frac{1}{\lambda} \bigg(\frac{e^{-\lambda y}}{\lambda}\bigg)\bigg|_0^\infty = \frac{1}{\lambda}.
         \end{align*}
         We now compute $E[Y^2]$ using integration by parts:
         \begin{align*}
            E[Y^2] &= \int_0^\infty y^2\lambda e^{-\lambda y} \dif y = -y^2e^{-\lambda y}\bigg|_0^\infty - \int_0^\infty -2ye^{-\lambda y} \dif y = \int_0^\infty 2ye^{-\lambda y} \dif y = \frac{2}{\lambda^2}.
        \end{align*}
        So, $\mathrm{Var}(Y) = \frac{2}{\lambda^2}-\frac{1}{\lambda^2} = \frac{1}{\lambda^2}$.\\
        
        \item Let $Y$ have PDF $f_Y(y) = 3(1-y)^2, \; 0 < y < 1$. Let $W = -5Y+12$. Find the variance and standard deviation of $W$.\\
        \textbf{Answer: }\\
        $\mathrm{Var}(W) = \mathrm{Var}(-5Y+12) = 25\mathrm{Var}(Y)$. Since we already found $\mathrm{Var}(Y) = \frac{3}{80}$, we can easily find that $\mathrm{Var}(W) = 25 \cdot \frac{3}{80} = \frac{15}{16}$.\\
    \end{enumerate}
    \textbf{Section 3.7 Problems}
    \begin{enumerate} \setcounter{enumi}{3}
        \item 3.7.1. If $p_{X,Y}(x,y) = cxy$ at the points $(1,1),\,(2,1),\,(2,2)$ and $(3,1)$, and equals $0$ elsewhere, find $c$.\\
        \textbf{Answer: }\\
        $p_{X,Y}(1,1) = c$, $p_{X,Y}(2,1) = 2c$, $p_{X,Y}(2,2) = 4c$, and $p_{X,Y}(3,1) = 3c$. Since $p_{X,Y}$ must sum to $1$, then $10c = 1 \implies c = \frac{1}{10}$.\\
        
        \item \textbf{(461 only)} 3.7.2. Let $X$ and $Y$ be two continuous random variables defined over the unit square. What does $c$ equal if $f_{X,Y}(x,y) = c(x^2+y^2)$?\\
        \textbf{Answer: }\\
        We use the fact that integrating over the entire unit square should equal 1:
        \begin{align*}
            &\int_0^1\int_0^1 c(x^2+y^2) \dif y \dif x = c \int_0^1 \bigg(x^2y+\frac{y^3}{3}\bigg)\bigg|_0^1 \dif x = c\int_0^1 x^2+\frac{1}{3} \dif x = c\bigg(\frac{x^3+x}{3}\bigg)\bigg|_0^1 = \frac{2c}{3} = 1\\
            &\implies c = \frac{2}{3}.
        \end{align*}
        
        \item 3.7.8. Consider the experiment of tossing a fair coin three times. Let $X$ denote the number of heads on the last flip, and let $Y$ denote the total number of heads on the three flips. Find $p_{X,Y}(x,y)$.\\
        \textbf{Answer: }\\
        The Sample Space $S = \{(H,H,H),(H,H,T),(H,T,H),(H,T,T),(T,H,H)(T,H,T),(T,T,H),(T,T,T)\}$. So then the discrete pdf $p_{X,Y}(x,y)$ can be displayed as follows, where the rows represent $X = $ \# of heads on the last flip and the columns represent $Y = $ \# of heads total:
        \begin{center}
            \begin{tabular}{|c|| c| c| c| c|}
                \hline
                & $Y=0$ & $Y=1$ & $Y=2$ & $Y=3$\\
                \hline\hline
                $X=0$ & $P(0,0) = \nicefrac{1}{8}$ & $P(0,1) = \nicefrac{2}{8}$ & $P(0,2) = \nicefrac{1}{8}$ & $P(0,3) = 0$\\
                \hline
                $X=1$ & $P(1,0) = 0$ & $P(1,1) = \nicefrac{1}{8}$ & $P(1,2) = \nicefrac{2}{8}$ & $P(1,3) = \nicefrac{1}{8}$\\
                \hline
                \end{tabular}
            \end{center}
        
        \item 3.7.11. Let $X$ and $Y$ have the joint pdf
            \[f_{X,Y}(x,y) = 2e^{-(x+y)}, \quad 0 < x < y, \quad 0 < y.\]
        Find $P(Y < 3X)$.\\
        \textbf{Answer: }
        \begin{align*}
            P(Y < 3X) &= \int_{-\infty}^\infty\int_{-\infty}^\infty f_{X,Y} (x,y) \dif y \dif x = \int_0^\infty\int_x^{3x} 2e^{-x} \cdot e^{-y} \dif y \dif x = \int_0^\infty 2e^{-x}(-e^{-y})\Big|_x^{3x} \dif x\\
            &= \int_0^\infty -2e^{-4x}+2e^{-2x} \dif x = -\frac{e^{-4x}}{2}-e^{-2x}\bigg|_0^\infty = \frac{1}{2}.
        \end{align*}
        
        \item 3.7.19. - Part (b) only. Find $f_X(x)$ and $f_Y(y)$: $f_{X,Y}(x,y) = \frac{3}{2}y^2, \; 0 \leq x \leq 2, \; 0 \leq y \leq 1$.\\
        \textbf{Answer: }
        \begin{align*}
            f_X(x) &= \int_{-\infty}^\infty f(x,y) \dif y = \int_0^1 \frac{3}{2}y^2 \dif y = \frac{3}{2}\bigg(\frac{y^3}{3}\bigg)\bigg|_0^1 = \frac{3}{6} = \frac{1}{2}.\\
            f_Y(y) &= \int_{-\infty}^\infty f(x,y) \dif x = \int_0^2 \frac{3}{2}y^2 \dif x = \frac{3}{2}(y^2x)\Big|_0^2 = \frac{3y^2}{2} \cdot 2 = 3y^2.
        \end{align*}
        
        \item 3.7.26. An urn contains twelve chips---four red, three black, and five white. A sample of size $4$ is to be drawn without replacement. Let $X$ denote the number of white chips in the sample, $Y$ the number of red. Find $F_{X,Y}(1,2)$.\\
        \textbf{Answer: }\\
        We first note that the order in which we draw the chips does not matter, so we can say that there are ${12 \choose 4} = 495$ ways for $4$ chips to be drawn. We then note that there are ${4 \choose y}$ ways to get $y$ red chips from the $4$ total red chips, and ${5 \choose x}$ ways to pick $x$ white chips from the $5$ total white chips. For the remaining black chips, we can figure out that there are ${3 \choose 4-(x+y)}$ ways for them to be chosen (since the sample size is $4$ and $x+y$ are the amount of chips already chosen). So then 
            \[p_{X,Y}(x,y) = \frac{{5 \choose x}{4 \choose y}{3 \choose 4-(x+y)}}{495}\]
        To find $F_{X,Y}(1,2)$, we need to sum all of the probabilities of the different random variables $X \leq 1$ and $Y \leq 2$ (However, we disregard the $P(X = 0, Y = 0)$ case otherwise there would not be enough chips in the sample size)
            \begin{align*}
                F_{X,Y}(1,2) &= P(X=0,Y=1) + P(X=0,Y=2) + P(X=1,Y=0) + P(X=1,Y=1) + P(X=1,Y=2)\\
                &= \frac{{5 \choose 0}{4 \choose 1}{3 \choose 3}}{495} + \frac{{5 \choose 0}{4 \choose 2}{3 \choose 2}}{495} + \frac{{5 \choose 1}{4 \choose 0}{3 \choose 3}}{495} + \frac{{5 \choose 1}{4 \choose 1}{3 \choose 2}}{495} + \frac{{5 \choose 1}{4 \choose 2}{3 \choose 1}}{495}\\
                &= \frac{(1 \cdot 4 \cdot 1)+(1 \cdot 6 \cdot 3)+(5 \cdot 1 \cdot 1)+(5 \cdot 4 \cdot 3)+(5 \cdot 6 \cdot 3)}{495} \approx 0.358.
            \end{align*}
            
        \item 3.7.28. - Part (a) only. Find $F_{X,Y}(x,y)$ given that $f_{X,Y}(x,y) = \frac{1}{2}, \; 0 \leq x \leq y \leq 2$.\\
        \textbf{Answer: }\\
        \begin{align*}
            F_{X,Y}(u,v) = \int_{-\infty}^u \int_ {-\infty}^v f_{X,Y}(x,y) \dif y \dif x = \int_0^u \int_ x^v \frac{1}{2} \dif y \dif x = \frac{1}{2}\int_0^u y\Big|_x^v \dif x = \frac{1}{2}\int_0^u v-x \dif x = \frac{1}{2}\Big(vx-\frac{x^2}{2}\Big)\Big|_0^u = \frac{uv}{2}-\frac{u^2}{4}.
        \end{align*}
        
        \item 3.7.38. Two fair dice are tossed. Let $X$ denote the number appearing on the first die and $Y$ the number on the second. Show that $X$ and $Y$ are independent.\\
        \textbf{Answer: }\\
        First note that the probablity for any number rolled is $\nicefrac{1}{6}$, i.e. $P_X(x) = P_Y(y) = \frac{1}{6},\; \forall X,Y$. Let each row be the random variable $X$ and each column be the random variable $Y$.
        \begin{center}
            \begin{tabular}{|c|| c| c| c| c| c| c|}
                \hline
                & $Y=1$ & $Y=2$ & $Y=3$ & $Y=4$ & $Y=5$ & $Y=6$\\
                \hline \hline
                $X=1$ & $2$ & $3$ & $4$ & $5$ & $6$ & $7$\\
                \hline
                $X=2$ & $3$ & $4$ & $5$ & $6$ & $7$ & $8$\\
                \hline
                $X=3$ & $4$ & $5$ & $6$ & $7$ & $8$ & $9$\\
                \hline
                $X=4$ & $5$ & $6$ & $7$ & $8$ & $9$ & $10$\\
                \hline
                $X=5$ & $6$ & $7$ & $8$ & $9$ & $10$ & $11$\\
                \hline
                $X=6$ & $7$ & $8$ & $9$ & $10$ & $11$ & $12$\\
                \hline
            \end{tabular}
        \end{center}
        Then by the table, $P_{X,Y}(x,y) = \frac{1}{36} = \frac{1}{6} \cdot \frac{1}{6} = P_X(x) \cdot P_Y(y)$. So $X$ and $Y$ are independent.\\
        
        \item 3.7.43. Suppose that random variables $X$ and $Y$ are independent with marginal pdfs $f_X(x) = 2x, \; 0 \leq x \leq 1$, and $f_Y(y) = 3y^2, \: 0 \leq y \leq 1$. Find $P(Y < X)$.\\
        \textbf{Answer: }\\
        We first find the joint pdf $f_{X,Y}(x,y)$ since $X$ and $Y$ are independent: $f_{X,Y}(x,y) = f_X(x) \cdot f_Y(y) = 2x \cdot 3y^2 = 6xy^2$. Now we integrate:
            \begin{align*}
                P(Y < X) &= \int_0^\infty\int_0^\infty f(x,y) \dif y \dif x = \int_0^1\int_0^x 6xy^2 \dif y \dif x = 2\int_0^1 xy^3\bigg|_0^x \dif x = 2\int_0^1 x^4 \dif x = 2\bigg(\frac{x^5}{5}\bigg)\bigg|_0^1 = \frac{2}{5}.
            \end{align*}
        
        \item \textbf{(461 only)} 3.7.44. Find the joint cdf of the independent random variables $X$ and $Y$, where $f_X(x) = \frac{x}{2}, \: 0 \leq x \leq 2$, and $f_Y(y) = 2y, \; 0 \leq y \leq 1$.\\
        \textbf{Answer: }\\
        We repeat the same process in \#12: $f_{X,Y}(x,y) = f_X(x) \cdot f_Y(y) = \frac{x}{2} \cdot 2y = xy$. Now we integrate:
            \begin{align*}
                F_{X,Y}(u,v) &= \int_{-\infty}^u\int_{-\infty}^v f(x,y) \dif y \dif x = \int_0^u\int_0^v xy \dif y \dif x = \frac{1}{2}\int_0^u xy^2\Big|_0^v \dif x = \frac{1}{2}\int_0^u xv^2 \dif x = \frac{1}{4}(x^2v^2)\Big|_0^u = \frac{u^2v^2}{4}.
            \end{align*}
    \end{enumerate}
\end{document}