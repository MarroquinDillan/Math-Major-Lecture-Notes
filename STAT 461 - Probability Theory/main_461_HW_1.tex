\documentclass{article}
\usepackage{amsmath}
\usepackage[margin=0.6in]{geometry}
\usepackage{graphicx}


\renewcommand{\familydefault}{\sfdefault}
\title{STAT 461.1002: Homework 1}
\author{Dillan Marroquin}
\date{September 4, 2020}
\begin{document}
\maketitle

    \begin{enumerate}
        \item Three dice are tossed, one red, one blue, and one green. What outcomes make up the event $A$ that the sum of the three faces showing equals 5?

        \textbf{Answer:} $A = \{1r+1b+3g$, $1r+2b+2g$, $1r+3b+1g$, $2r+1b+2g$, $2r+2b+1g$, $3r+1b+1g\}$\\

        \item Two darts are thrown at the following target:
            \begin{enumerate}
                \item  Let $(u, v)$ denote the outcome that the first dart lands in region $u$ and the second dart, in region $v$. List the sample space of $(u, v)$ pairs.
                
                \textbf{Answer:} $S = \{(1,1)$, $(1,2)$, $(1,4)$, $(2,1)$, $(2,2)$, $(2,4)$, $(4,1)$, $(4,2)$, $(4,4)\}$\\

                \item List the outcomes in the sample space of sums, $u+v$.
                
                \textbf{Answer:} $S = \{2, 3, 4, 5, 6, 8\}$\\
            \end{enumerate}

        \item Define $A = \{x : 0 \leq x \leq 1 \}$, $B = \{ x : 0 \leq x \leq 3 \}$, and $C = \{ x : -1 \leq x \leq 2 \}$. Draw diagrams showing each of the following sets of points:
            \begin{enumerate}
                \item $A^C \cap B \cap C$ \\
                {\includegraphics[scale=.05, angle = 90] {3a.jpg}} \\


                \item $[(A \cup B) \cap C^C]^C$\\
                {\includegraphics[scale=.05, angle = 90] {3b.jpg}}\\
                
            \end{enumerate}
            
        \item Suppose that three events $A$, $B$, and $C$ are defined on a sample space $S$. Use the union, intersection, and complement operations to represent each of the following events:
            \begin{enumerate}
                \item exactly one event occurs
                
                    \textbf{Answer:} $(A \cup B \cup C) \backslash (A \cap B \cap C)$\\
                    
                \item exactly two events occur
                
                    \textbf{Answer:} $[(A \cap B) \backslash C] \cup [(A \cap C) \backslash B] \cup [(B \cap C) \backslash A]$\\
            \end{enumerate}
            
        \item Let $A$ and $B$ be any two events defined on $S$. Suppose that $P(A) = 0.4$, $P(B) = 0.5$, and $P(A \cap B) = 0.1$. What is the probability that $A$ or $B$ but not both occur?
        
            \textbf{Answer:} The probability of $A$ or $B$ but not both is the same as saying $P(A \cup B) - P(A \cap B)) = 0.9 - 0.1 = 0.8$.\\
    
        \item Suppose that three fair dice are tossed. Let $A_i$ be the event that a 6 shows on the $i^{\text{th}}$ die, $i=1, 2, 3$. Does $P(A_1 \cup A_2 \cup A_3) = \frac{1}{2}$? Explain.
            
            \textbf{Answer:} Yes, $P(A_1 \cup A_2 \cup A_3) = \frac{1}{2}$. This is because events $A_1$, $A_2$, and $A_3$ are mutually exclusive events, so
            \[P(A_1 \cup A_2 \cup A_3) = P(A) + P(B) + P(C) = \frac{1}{6} + \frac{1}{6} + \frac{1}{6} = \frac{1}{2}.\]
            
         \item An urn contains twenty-four chips, numbered 1 through 24. One is drawn at random. Let $A$ be the event that the number is divisible by 2 and let $B$ be the event that the number is divisible by 3. Find $P(A \cup B)$.
        
            \textbf{Answer:} $P(A) = \frac{1}{2}$ (obviously) and $P(B) = \frac{1}{3}$ (obviously), so $P(A \cup B) = \frac{1}{2} + \frac{1}{3} - \frac{1}{6} = \frac{2}{3}$.\\
         
         \item Three events $A$, $B$, and $C$ are defined on a sample space, $S$. Given that $P(A)=0.2$, $P(B)=0.1$, and $P(C)=0.3$, what is the smallest possible value for $P[(A \cup B \cup C)^C]$?
    
            \textbf{Answer:} Smallest possible value of $P[(A \cup B \cup C)^C = 0$.\\
            
        \item If $P(A) = \frac{1}{2}$ and $P(B^C) = \frac{1}{3}$, can $A$ and $B$ be disjoint? Explain.
        
            \textbf{Answer:} No, they cannot be disjoint. If we suppose $A$ and $B$ are disjoint, then $P(B)^C$ would also include $P(A)$ in its entirety, so $P(B)^C \geq P(A)$. However, this leads to a contradiction as $P(B)^C = \frac{1}{3}$ and $P(A) = \frac{1}{2}$, so $P(B)^C < P(A)$.\\
            
         \item \textbf{(461 only)} Express the following probability in terms of $P(A)$, $P(B)$, and $P(A \cap B)$: $P(A^C \cap (A \cup B))$.
         
            \textbf{Answer:} $P(A^C \cap (A \cup B)) = (1 - P(A)) \cap (P(A) + P(B) - P(A \cap B))$.
    \end{enumerate}


\end{document}