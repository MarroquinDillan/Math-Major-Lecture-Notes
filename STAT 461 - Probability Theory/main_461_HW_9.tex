\documentclass{article}
\usepackage{amsmath}
\usepackage{commath}
\usepackage{units}
\usepackage{kpfonts}
\usepackage[margin=0.6in]{geometry}
\usepackage{multicol}

\begin{document}
    \noindent STAT 461.1002: Homework 9\\
    Dillan Marroquin\\
    Monday December 7\\

    \noindent \textbf{Chapter 4 Problems}
    \begin{enumerate}
        \item 4.2.2 A medical study recently documented that 905 mistakes were made among the 289,411 prescriptions written during one year at a large metropolitan teaching hospital. Suppose a patient is admitted with a condition serious enough to warrant ten different prescriptions. Approximate the probability that at least one will contain an error.\\
        \textbf{Answer: } Let $n=10$ and $p = \frac{905}{289411}$. Then $\lambda = 10\cdot \frac{905}{289411} \approx 0.03127$ and
            \begin{align*}
                P(Y=0) &= \frac{e^{-0.03127}(0.03127)^0}{0!} \approx 0.9692\\
                \implies P(Y \geq 1) &= 1-0.9692 = 0.0308. 
            \end{align*}
        
        \item 4.2.17 In a new fiber-optic communication system, transmission errors occur at the rate of 1.5 per ten seconds. What is the probability that more than two errors will occur during the next half-minute?\\
        \textbf{Answer: }\\In this case, let $n=30$ (30 seconds in 1 half-minute) and let $p=\frac{1.5}{10}$ (1.5 errors every 10 seconds). We must evaluate $P(X>2) = 1-P(X\leq 2) = 1-[P(X=0)+P(X=1)+P(X=2)$
            \begin{align*}
                P(X=0) &= \frac{e^{-\frac{9}{2}}\big(\frac{9}{2}\big)^0}{0!}       &       P(X=1) &= \frac{e^{-\frac{9}{2}}\big(\frac{9}{2}\big)^1}{1!}       &       P(X=2) &= \frac{e^{-\frac{9}{2}}\big(\frac{9}{2}\big)^2}{2!}\\
                &\approx 0.0111        &       &\approx 0.05
            \end{align*}
            \begin{align*}
                \implies P(X > 2) = 1-P(X \leq 2) = 1-(0.0111+0.05+0.1125) = 0.8264.
            \end{align*}
        
        \item \textbf{(461 only) }4.2.22 Find $P(X=4)$ if the random variable $X$ has a Poisson distribution such that $P(X=1) = P(X=2)$.\\
        \textbf{Answer: }Since we know that $X$ follows a Poisson distribution and that $P(X=1)=P(X=2)$, then
            \begin{align*}
                P(X=1) &= P(X=2)\\
                e^{-\lambda}\lambda &= \frac{e^{-\lambda} \lambda^2}{2}\\
                2\lambda e^{-\lambda} &= e^{-\lambda}\lambda^2\\
                2\lambda = \lambda^2 &\iff 2=\lambda\\
                \implies P(X=4) &= \frac{e^{-2}2^4}{4!} \approx 0.0902.
            \end{align*}
        
        \item 4.3.2 (b,c)  Let $Z$ be a standard normal random variable. Use Appendix Table A.1 to find the numerical value for each of the following probabilities. Show each of your answers as an area under $f_Z(z)$.
        \begin{enumerate}
            \item $P(-0.64 \leq Z < -0.11)$\\
            \textbf{Answer: }$P(-0.64 \leq Z < -0.11) = 0.4562-0.2611 = 0.1951$.
            
            \item$ P(Z > -1.06)$\\
            \textbf{Answer: }$P(Z > -1.06) = 1-0.1446 = 0.8554$.
        \end{enumerate}
        
        \item 4.3.5 (d) Assume that the random variable $Z$ is described by a standard normal curve $f_Z(z)$. What values of $z$ makes $P(-z < Z < z) = 0.80$ true?\\
        \textbf{Answer: }
            \begin{align*}
                P(-z < Z < z) &= 0.80\\
                P(Z < z) - (1-P(Z < z)) &= 0.80    \qquad \textit{Since symmetric}\\
                2P(Z < z) -1 &= 0.80\\
                \iff P(Z < z) &= 0.90\\
            \end{align*}
        According to an online calculator, $z=1.282$.\\
        
        \item 4.3.34 Let $Y_1,Y_2, \ldots, Y_n$ be a random sample from a normal distribution where the mean is 2 and the variance is 4. How large must $n$ be in order that
            \[P(1.9 \leq \overline{Y} \leq 2.1) \geq 0.99\]
        \textbf{Answer: }\\
        
        \item 4.4.7 Let $Y$ be an exponential random variable, where $f_Y(y) = \lambda e^{-\lambda y}, \, 0 \leq y$. For any positive integer $n$, show that $P(n \leq Y \leq n+1) = e^{-\lambda n}(1-e^{-\lambda})$. Note that if $p = 1-e^{-\lambda}$, the "discrete" version of the exponential pdf is the geometric pdf.\\
        \textbf{Answer: }We integrate
            \begin{align*}
                P(n \leq Y \leq n+1) &= \int_n^{n+1} \lambda e^{-\lambda y} \dif y\\
                &= -e^{-\lambda y}\Big|_n^{n+1}\\
                &= -e^{-\lambda n-\lambda} + e^{-\lambda n} = -e^{-\lambda n} \cdot e^{-\lambda} + e^{-\lambda n} = e^{-\lambda n}(1-e^{-\lambda}).
            \end{align*}
        
        \item 4.5.6 Let the random variable $X$ denote the number of trials in excess of $r$ that are required to achieve the $r$th success in a series of independent trials, where $p$ is the probability of success at any given trial. Show that
            \[p_X(k) = {k+r-1 \choose k} p^r(1-p)^k, \quad k=0,1,2,\ldots\]
        (\textit{Note:} This particular formula for $p_X(k)$ is often used in place of Equation 4.5.1 as the definition of the pdf for a negative binomial random variable.)\\
        To clarify, "the number of trials in excess of $r$" is just the number of failures before the $r$th success.\\
        \textbf{Answer: }Let $k+r-1$ mean the number of $k$ failures needed until the required success, $r$. Note that each trial is independent. Then we achieve 
            \begin{align*}
                P(X=k) &= {k+r-1 \choose k} \cdot P(\text{$k$ fails AND $r-1$ successful trials}) \cdot P(\text{Succeeding})\\
                &= P(\text{$k$ fails})\cdot P(\text{$r-1$ successful trials}) \cdot P(\text{Succeeding})\\
                &= p_X(k) = {k+r-1 \choose k} p^r(1-p)^k.
            \end{align*}
        
        \item \textbf{Memoryless property of exponential waiting times: }Let $T$ be an exponential RV with rate $r$, i.e.,
            \[f_T(t) = re^{-rt}    \quad\text{and}\quad F_T(t) = P(T \leq t) = 1-e^{-rt},  \quad r \geq 0.\]
        $T$  represents the waiting time until an event occurs. Suppose 10 minutes go by, and no event occurs. What is the distribution of the waiting time after the 10 minute mark?\\
        To answer this, first define $U$ to be the waiting time after 10 minutes, where $F_U(t) = P(U \leq t) = P(T \leq 10+t|T \geq 10)$. Use the CDF of $T$ and the definition of conditional probability to show that $U$ is exponential with rate $r$. That is, show that
            \[P(U \leq t) = 1-e^{-rt}.\]
        \textbf{Answer: }Since the events are independent, let's apply the definition of conditional probability:
            \begin{align*}
                P(U \leq t) &= P(T \leq 10+t|T \geq 10)\\
                &= \frac{\text{$P(T \leq 10+t)$ and $P(T \geq 10)$}}{P(T \geq 10)}\\
                &= \frac{P(T \leq 10+t) \cdot P(T \geq 10)}{P(T \geq 10)}\\
                &= P(T \leq 10+t)\\
                &= 1-e^{-rt}.
            \end{align*}
        
        \item Recall Example 3.12.3 (also done in class) which shows that the MGF for $T$, an exponential RV with rate $r$, is $M_T(t) = \frac{r}{r-t}$ for $t < r$. See Theorem 4.6.5 which shows that the MGF for $Y$, a gamma RV with shape parameter $n$ and rate $r$, is $M_Y(t) = \frac{r^n}{(r-t)^n}$. Use the properties of moment generating functions spelled out in Theorems 3.12.2 and 3.12.3 to show that the sum of $n$ \textit{i.i.d.} exponentials (with rate $r$) is a gamma distributed RV. \\
        \textbf{Answer: }First, let $U=\sum_{k=0}^n T_k$. Then we have
            \begin{align*}
                M_U(t) &= (M_T(t))^n\\
                &= \bigg(\frac{r}{r-t}\bigg)^n\\
                &= \frac{r^n}{(r-t)^n} = M_Y(t).
            \end{align*}
        Since the MGF for $U$ is equal to the MGF for a gamma RV, the sum of $n$ MGFs for an exponential RV is equal to the MGF for a gamma RV.\\
        
        \item Let $X_1, \ldots, X_n$  be independent exponential RVs, each with rate $r_i$ (independent but not identically distributed because the $r_i$’s are allowed to be different).
        \begin{enumerate}
            \item Using the MGFs from Problem 10 above, find $M_{X_i} (0), \, M_{X_i}' (0)$ and $E[X_i]$.\\
            \textbf{Answer: }
                \begin{align*}
                    M_{X_i}(0) &= \frac{r_i}{r_i-0}     &       M_{X_i}'(0) &= \dod{}{t}\frac{r_i}{r_i-t}     &       E[X_i] &= M_{X_i}'(0)\\
                    &= \frac{r_i}{r_i} = 1.     &       &= \frac{r_i}{(r_i-t)^2} = \frac{1}{r_i}.      &         &= \frac{1}{r_i}.
                \end{align*}
            
            \item Let $Y = \sum_{i=1}^n X_i$. Using the properties of MGFs, and the generalized product rule for the product of $n$ differentiable functions $G_i(t)$,
                \[\od{}{t}\bigg(\prod_{i=1}^n G_i(t)\bigg) = \sum_{i=1}^n\bigg(G_i'(t)\prod_{j\neq i}  G_j(t) \bigg)\]
            find $M_Y'(0)$ to confirm the obvious relationship $E[Y] = \sum_{i=1}^n E[X_i]$.\\
            \textbf{Answer: }We will use the same method we used in \#10.
                \begin{align*}
                    M_Y(t) &= (M_{X_i}(t))^n\\
                    &= \bigg(\frac{r}{r-t}\bigg)^n\\
                    &= \frac{r^n}{(r-t)^n}.
                \end{align*}
            So $\od{}{t}M_Y = r^nn(r-t)^{-n-1}$ and $M_Y'(0) = \frac{1}{r} = E[X_i]$.
        \end{enumerate}
        
    \end{enumerate}
\end{document}