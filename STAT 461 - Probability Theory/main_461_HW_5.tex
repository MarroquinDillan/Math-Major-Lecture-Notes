\documentclass{article}
\usepackage{amsmath}
\usepackage{commath}
\usepackage{units}
\usepackage{kpfonts}
\usepackage[margin=0.6in]{geometry}
\usepackage{pgfplots}
\usepackage{multicol}

\pgfplotsset{compat=1.17}

\begin{document}
    \noindent STAT 461.1002: Homework 5\\
    Dillan Marroquin\\
    16 October, 2020\\

    \textbf{Section 3.3 Problems}
    \begin{enumerate}
        \item Consider a random variable $X$ such that $P(X=-1) = 0.4$, $P(X=0)=0.3$, and $P(X=2)=0.3$. Find the cumulative distribution function (CDF) of $X$ (Hint: make a table) and sketch a graph of the CDF.\\
        \textbf{Answer: }
            \begin{multicols}{2}
                \begin{center}
                    \begin{tabular}{|c| c| c|}
                        \hline
                        $k$ & $P_X(k)$ (PMF) & $F_X(k)$ (CDF)\\
                        \hline
                        $-1$ & $0.4$ & $P(X \leq -1) = 0.4$\\ 
                        \hline
                        $0$ & $0.3$ & $P(X \leq 0) = 0.7$\\
                        \hline
                        $2$ & $0.3$ & $P(X \leq 2) = 1.0$\\
                        \hline
                    \end{tabular}
                \end{center}
                
                \columnbreak
                
                \begin{tikzpicture}[scale=0.7]
                    \begin{axis}[ytick={0,.2,.4,.6,.8,1}, axis lines = left, xlabel = $k$, xmin=-2, xmax=3, ylabel = $F_X(k)$, ymin=0, ymax=1.1]
                    
                \draw[dashed] (0,0.4)--(0,0.7);
                \draw[dashed] (2,0.7)--(2,1);
                        \addplot[domain=-1:0] {0.4};
                        \addplot[domain=0:2] {0.7};
                        \addplot[domain=2:3] {1};
                        \addplot[mark=*,fill=white] coordinates {(0,0.4)};
                        \addplot[mark=*,fill=white] coordinates {(2,0.7)};
                    \end{axis}
                \end{tikzpicture}
            \end{multicols}

        \item 3.3.11 Suppose $X$ is a binomial random variable with $n=4$ and $p=\frac{2}{3}$. What is the pdf of $2X+1$?\\
        \textbf{Answer: }We know $P(X=k) = {4 \choose k}\big(\frac{2}{3}\big)^4$ for $k=0,1,2,3,4$. So $P(Y=k) = P\big(X=\frac{k-1}{2}\big) = {4 \choose \frac{k-1}{2}}\big(\frac{2}{3}\big)^4$ for $k=1,3,5,7,9$.
            \begin{multicols}{3}
                \begin{align*}
                    P(Y=1) &= {4 \choose 0}\bigg(\frac{2}{3}\bigg)^4 = \frac{16}{81}\\
                    P(Y=3) &= {4 \choose 1}\bigg(\frac{2}{3}\bigg)^4 = \frac{64}{81}
                \end{align*}
                \columnbreak
                \begin{align*}
                    P(Y=5) &= {4 \choose 2}\bigg(\frac{2}{3}\bigg)^4 = \frac{32}{27}\\
                    P(Y=7) &= {4 \choose 3}\bigg(\frac{2}{3}\bigg)^4 = \frac{64}{81}\\
                \end{align*}
                \columnbreak
                \begin{align*}
                    P(Y=9) &= {4 \choose 4}\bigg(\frac{2}{3}\bigg)^4 = \frac{16}{81}
                \end{align*}
            \end{multicols}
    \end{enumerate}
    
    \textbf{Section 3.4 Problems}
    \begin{enumerate}\setcounter{enumi}{2}
        \item 3.4.2 For the random variable $Y$ with pdf $f_Y(y) = \frac{2}{3}+\frac{2}{3}y$, $0\leq y \leq 1$, find $P(\frac{3}{4} \leq Y \leq 1)$.\\
        \textbf{Answer: }Directly apply formula for continuous pdf:
            \begin{align*}
                P(\nicefrac{3}{4} \leq Y \leq 1) &= \int_{\nicefrac{3}{4}}^1 \frac{2}{3}y+\frac{2}{3}\dif y\\
                &= \eval{\frac{1}{3}y^2+\frac{2}{3}y}_\frac{3}{4}^1 = 1-\Big(\frac{9}{48}+\frac{1}{2}\Big) = \frac{5}{16}.
            \end{align*}
        
        \item 3.4.6 Let $n$ be a positive integer. Show that $f_Y(y) = (n+2)(n+1)y^n(1-y)$, $0 \leq y \leq 1$, is a pdf.\\
        \textbf{Answer: }The easiest way to show that this is a pdf is by seeing if the integral over $\mathbb{R}$ on $f_Y(y) = 1$:
            \begin{align*}
                \int_{-\infty}^\infty f(y) \dif y &= \int_0^1 (n+2)(n+1)y^n(1-y)\dif y\\
                &= (n+2)(n+1)\int_0^1 y^n-y^{1+n}\dif y\\
                &= (n+2)(n+1)\Bigg(\eval{\frac{y^{n+1}}{n+1}-\frac{y^{n+2}}{n+2}}_0^1\Bigg)\\
                &= (n+2)(n+1)\bigg(\frac{n+2-n-1}{(n+2)(n+1)}\bigg) = (n+2)(n+1)\bigg(\frac{1}{(n+2)(n+1)}\bigg) = 1
            \end{align*}
        So, $f_Y(y)$ is a pdf.\\
        
        \item 3.4.7 Find the cdf for the random variable $Y$ given in Question 3.4.1 ($f_Y(y) = 4y^3$, $0 \leq y \leq 1$). Calculate $P(0 \leq Y \leq \frac{1}{2})$ using $F_Y(y)$.\\
        \textbf{Answer: }
            \begin{align*}
                F_Y(y) = \int_0^y f_Y(y) \dif y = \int_0^y 4y^3 \dif y = \eval{y^4}_0^y = y^4.
            \end{align*}
        So, $P(0 \leq Y \leq \nicefrac{1}{2}) = (\nicefrac{1}{2})^4 = \nicefrac{1}{16}$.\\
        
        \item\textbf{461 only} 3.4.17 Suppose that $f_Y(y)$ is a continuous and symmetric pdf, where \textit{symmetry} is the property that $f_Y(y) = f_Y(-y)$ for all $y$. Show that $P(-a \leq Y \leq a) = 2F_Y(a)-1$.\\
        \textbf{Answer: }
            \begin{align*}
                P(-a \leq Y \leq a) &= \int_{-\infty}^af_Y(y)\dif y + \int_{-a}^\infty f_Y(y)\dif y-1\\
                &= \int_{-\infty}^af_Y(y)\dif y + \int_{-\infty}^a f_Y(y)\dif y-1 & \text{Because $f_Y(y)$ is symmetric}\\
                &= 2\int_{-\infty}^a f_Y(y)\dif y - 1 & \text{Combined the integrals}\\
                &= 2F_Y(a)-1.
            \end{align*}
    \end{enumerate}
    
    \textbf{Section 3.5 Problems}
    \begin{enumerate}\setcounter{enumi}{6}
        \item Find the expected value of the random variable $X$ defined in Problem 1 ($X$ such that $P(X=-1) = 0.4$, $P(X=0)=0.3$, and $P(X=2)=0.3$).\\
        \textbf{Answer: }$E[X] = -1(0.4)+0(0.3)+2(0.3) = \nicefrac{1}{5}$\\
                
        \item 3.5.6 A manufacturer has one hundred memory chips in stock, $4\%$ of which are likely to be defective (based on past experience). A random sample of twenty chips is selected and shipped to a factory that assembles laptops. Let $X$ denote the number of computers that receive faulty memory chips. Find $E(X)$.\\
        \textbf{Answer: }$p = .04$, $n=20$, so $E[X] = (.04)(20) = \nicefrac{4}{5}$.\\
        
        \item 3.5.10 Let the random variable $Y$ have the uniform distribution over $[a,b]$; that is, $f_Y(y) = \frac{1}{b-a}$ for $a \leq y \leq b$. Find $E(Y)$ using Definition 3.5.1. Also, deduce the value of $E(Y)$, knowing that the expected value is the center of gravity of $f_Y(y)$.\\
        \textbf{Answer: }Apply the definition directly:
            \begin{align*}
                 E[Y] = \mu &= \int_{-\infty}^\infty y \cdot f_Yy \dif y\\ 
                &= \int_a^b y \cdot \frac{1}{b-a} \dif y = \frac{1}{b-a}\Bigg(\eval{\frac{y^2}{2}}_a^b\Bigg) = \frac{1}{b-a}\bigg(\frac{b^2-a^2}{2}\bigg) = \frac{b+a}{2}.
            \end{align*}
        Alternatively, we know by deduction that the center of gravity is the midpoint of the given interval $[a,b]$. So, $E[Y] = \nicefrac{b+a}{2}$
        
        \item 3.5.12 Show that 
            \[f_Y(y) = \frac{1}{y^2}, \;  y\geq 1\]
        is a valid pdf but that $Y$ does not have a finite expected value.\\
        \textbf{Answer: }
            \begin{align*}
                P(1 \leq Y < \infty) &= \int_1^\infty \frac{1}{y^2}\dif y = \eval{-\frac{1}{y}}_1^\infty = 0 -(-1) = 1. 
            \end{align*}
        Because $P(1 \leq Y < \infty) = 1$, the pdf is valid. However, there is no finite expected value because
            \begin{align*}
                E[Y] &= \int_1^\infty y \cdot \frac{1}{y^2} \dif y = \eval{\ln{|y|}}_1^\infty = \infty.  
            \end{align*}
        
        \item 3.5.33 Grades on the last Economics 301 exam were not very good. Graphed, their distribution had a shape similar to the pdf
            \[f_Y(y) = \frac{1}{5000}(100-y), \; 0 \leq y \leq 100\]
        As a way of "curving" the results, the professor announces that he will replace each person's grade, $Y$, with a new grade, $g(Y)$, where $g(Y) = 10\sqrt{Y}$. Will the professor's strategy be successful in raising the class average above $60$?\\
        \textbf{Answer: }
            \begin{align*}
                E[g(Y)] = \int_{-\infty}^\infty g(y)f_Y(y) \dif y &=  \frac{1}{500}\bigg(\int_0^{100} 100\sqrt{y} - \sqrt{y^3}\dif y\bigg)\\
                &= \frac{1}{500}\Bigg(\eval{\frac{200\sqrt{y^3}}{3} - \frac{2\sqrt{y^5}}{5}}_0^{100}\Bigg) = \frac{160}{3} \approx 53.\overline{3}.
            \end{align*}
        No, this will not raise the average above $60$.
    \end{enumerate}
\end{document}