\documentclass{article}
\usepackage{amsmath}
\usepackage{commath}
\usepackage{units}
\usepackage{kpfonts}
\usepackage[margin=0.6in]{geometry}
\usepackage{multicol}

\begin{document}
    \noindent STAT 461.1002: Homework 8\\
    Dillan Marroquin\\
    19 November, 2020\\

    \noindent \textbf{Section 3.11 Problems}
    \begin{enumerate}
        \item 3.11.4. Five cards are dealt from a standard poker deck. Let $X$ be the number of aces received, and $Y$ the number of kings. Compute $P(X = 2|Y = 2)$.\\
        \textbf{Answer: }We first recognize that $p_Y(y)$ and $P_{X,Y}(x,y)$ both follow hypergeometric distributions.\\Let $w = \text{\# of kings in a 52 deck} = 4$, $N = \text{\# of cards in a standard deck} = 52$, $n = \text{\# of cards dealt} = 5$. Then
        \begin{align*}
            p_Y(2) &= \frac{\displaystyle{4 \choose y=2}{48 \choose 5-(y=2)}}{\displaystyle{52 \choose 5}} = \frac{6 \cdot 17296}{2598960} = \frac{2162}{54145}\\
            p_{X,Y}(2,2) &= \frac{\displaystyle{4 \choose x=2}{4 \choose y=2}{44 \choose 5-(2)-(2)}}{\displaystyle{52 \choose 5}} = \frac{6^2(44)}{2598960} = \frac{33}{54145}\\
            \therefore P(X = 2|Y = 2) &= \frac{p_{X,Y}(2,2)}{p_Y(2)} = \frac{33/54145}{2162/54145} = \frac{33}{2162} \approx 0.0153.
        \end{align*}
            
        \item 3.11.5. Given that two discrete random variables $X$ and $Y$ follow the joint pdf $p_{X,Y} (x, y) = k(x + y)$, for $x = 1, 2, 3$ and $y = 1, 2, 3$,
        \begin{enumerate}
            \item Find $k$.\\
            \textbf{Answer: } We use the fact that the sum of all values of $x$ and $y$ in our pdf must equal 1:
            \begin{align*}
                1 &= \sum_{x=1}^3 \sum_{y=1}^3 k(x+y) = k \sum_{x=1}^3 (x+1)+(x+2)+(x+3) = k\sum_{x=1}^3 3x+6\\
                &= k[(3+6)+(6+6)+(9+6)] = 36k
                \implies k = \frac{1}{36}.
            \end{align*}
            
            \item Evaluate $p_{Y|x}(1)$ for all values of $x$ for which $p_X(x) > 0$.\\
            \textbf{Answer: }We use the fact that the marginal pdf of $x$ is equal to the sum of all probabilities of $y$. So,
                \begin{align*}
                    p_X(x) &= \sum_{y=1}^3 \frac{1}{36} (x+y) = \frac{1}{36}[(x+1)+(x+2)+(x+3)] = \frac{1}{36}(3x+6) = \frac{x+2}{12}.\\
                    \therefore p_{Y|x}(1) &= \frac{p_{X,Y} (x,1)}{p_X(x)} = \frac{\frac{x+1}{36}}{\frac{x+2}{12}} = \frac{x+1}{3x+6}\qquad \text{Now plug in $x=1,2,3$ separately\ldots}
                \end{align*}
                \begin{align*}
                    p_{Y|1}(1) &= \frac{(1)+1}{3(1)+6} = \frac{2}{9} &   p_{Y|2}(1) &= \frac{(2)+1}{3(2)+6} = \frac{1}{4}  &   p_{Y|3}(1) &= \frac{(3)+1}{3(3)} = \frac{2}{9}.
                \end{align*}
        \end{enumerate}

        \item 3.11.9. Let $X$ and $Y$ be independent Poisson random variables where $p_x(k) = e^{-\lambda} \frac{\lambda^k}{k!}$ and $p_Y (k) = e^{-\mu} \frac{\mu^k}{k!}$ for $k = 0, 1, \ldots$. Show that the conditional pdf of $X$ given that $X + Y = n$ is binomial with parameters $n$ and $\frac{\lambda}{\lambda+\mu}$. (Hint: See Question 3.8.3.)\\
        \textbf{Answer: }We want to show that $P(X=k|X+Y)$, the conditional pdf of $X$ given $X+Y=n$, is binomial. We use the fact that $X$ and $Y$ are independent and the definition of conditional probability to assist us.
            \begin{align*}
                P(X=k|X+Y) &= \frac{P(X=k) \cdot P(X+Y)}{P(X+Y)} = \frac{P(X=k) \cdot P(X+Y)}{\sum_{i=0}^n P(X = i) \cdot P(Y=n-i)}\\
                &= \frac{\frac{e^{-\lambda}\lambda^k}{k!} \cdot \frac{e^{-\mu}\mu^{n-k}}{(n-k)!}}{\sum_{i=0}^n  \frac{e^{-\lambda}\lambda^i}{i!} \cdot  \frac{e^{-\mu}\mu^{n-i}}{(n-i)!}} = \frac{\frac{\lambda^k \mu^{n-k}}{k!(n-k)!}}{\frac{1}{n!} \sum_{i=0}^n \frac{n!\lambda^i\mu^{n-i}}{i!(n-i)!}} = \cdots = \frac{n!}{k!(n-k)!}\bigg(\frac{\lambda}{\lambda+\mu}\bigg)^k\bigg(\frac{\mu}{\lambda+\mu}\bigg)^{n-k}
            \end{align*}
        Observe that this is binomial, with parameters $\big(n, \frac{\lambda}{\lambda+\mu}\big)$.\\
        
        \item 3.11.16. Suppose that $X$ and $Y$ are distributed according to the joint pdf
            \[f_{X,Y}(x,y) = \frac{2}{5} \cdot (2x+3y), \quad 0 \leq x \leq 1, \quad 0 \leq y \leq 1\]
        Find
        \begin{enumerate}
            \item $f_X(x)$.\\
            \textbf{Answer: }
                \begin{align*}
                    f_X(x) &= \int_0^1 f_{X,Y}(x,y) \dif y = \frac{2}{5} \int_0^1 2x+3y \dif y = \frac{2}{5}\bigg(2xy+\frac{3y^2}{2}\bigg)\bigg|_0^1 = \frac{4x+3}{5}.
                \end{align*}
            
            \item $f_{Y|x}(y)$.\\
            \textbf{Answer: }
                \begin{align*}
                    f_{Y|x}(y) &= \frac{f_{X,Y}(x,y)}{f_X(x)} = \frac{\frac{4x+6y}{5}}{\frac{4x+3}{5}} = \frac{4x+6y}{4x+3}.
                \end{align*}
            
            \item $P\big(\frac{1}{4} \leq Y \leq \frac{3}{4}|X = \frac{1}{2}\big)$.\\
            \textbf{Answer: }
                \begin{align*}
                    P(\nicefrac{1}{4} \leq Y \leq \nicefrac{3}{4} \;| \; X = \nicefrac{1}{2}) &= \int_{\nicefrac{1}{4}}^{\nicefrac{3}{4}} f_{Y|\nicefrac{1}{2}}(y) \dif y = \int_{\nicefrac{1}{4}}^{\nicefrac{3}{4}} \frac{4(\frac{1}{2})+6y}{4(\frac{1}{2})+3} \dif y = \frac{2}{5} \int_{\nicefrac{1}{4}}^{\nicefrac{3}{4}} 1+3y \dif y = \frac{2}{5}\bigg(y+\frac{3y^2}{2}\bigg)\bigg|_{\nicefrac{1}{4}}^{\nicefrac{3}{4}}\\
                    &= \cdots = \frac{1}{2}.
                \end{align*}
            
            \item $E(Y|x)$\\
            \textbf{Answer: }
                \begin{align*}
                    E[Y|x] &= \int_0^1 y \cdot f_{Y|x}(y) \dif y = \int_0^1 \frac{4xy+6y^2}{4x+3} \dif y = \frac{2}{4x+3}\int_0^1 3y^2+2xy \dif y = \frac{2}{4x+3}\bigg(y^3+xy^2\bigg)\bigg|_0^1\\
                    &= \cdots = \frac{2x+2}{4x+3}.
                \end{align*}
        \end{enumerate}

        \item \textbf{(461 only) }3.11.17. If $X$ and $Y$ have the joint pdf
            \[f_{X,Y}(x,y) = 2, \quad 0 \leq x < y \leq 1\]
        find $P\big(0 < X < \frac{1}{2}|Y=\frac{3}{4}\big)$.\\
        \textbf{Answer: } We first find $f_Y(y)$, then use this to compute $f_{X|y}(x)$.
            \begin{align*}
                f_Y(y) &= \int_0^y 2 \dif x = 2x\Big|_0^y = 2y.\\
                f_{X|y}(x) &= \frac{f_{X,Y}(x,y)}{f_Y(y)} = \frac{2}{2y} = \frac{1}{y}.\\
                P(0<X<\nicefrac{1}{2}|Y=\nicefrac{3}{4}) &= \int_0^\frac{1}{2} f_{X|\nicefrac{3}{4}} (x) \dif x = \int_0^\frac{1}{2} \frac{4}{3} \dif x = \frac{4x}{3}\bigg|_0^\frac{1}{2} = \frac{2}{3}.
            \end{align*}
    \end{enumerate}

    \noindent \textbf{Section 3.12 Problems}
    \begin{enumerate}\setcounter{enumi}{5}
        \item 3.12.3. (Hint: See Examples in section 3.12) Find $E[e^{3X}]$ if $X$ is a binomial random variable with $n = 10$ and $p = \frac{1}{3}$.\\
        \textbf{Answer: }We apply the formula for the MGF of a binomial distribution, where $n=10, \, p=\nicefrac{1}{3}$.
            \begin{align*}
                M_X(t) &= (1-p+pe^t)^n = \bigg(1-\frac{1}{3}+\frac{1}{3}e^t\bigg)^{10} = \bigg(\frac{2+e^t}{3}\bigg)^{10}\\
                M_X(3) &= \bigg(\frac{2+e^3}{3}\bigg)^{10} \approx 467,591,999.
            \end{align*}

        \item 3.12.4. (Hint: Carefully compare to textbook example 3.12.1) Find the moment-generating function for the discrete random variable $X$ whose probability function is given by
            \[p_X(k) = \bigg(\frac{3}{4}\bigg)^k\bigg(\frac{1}{4}\bigg), \quad k=0,1,2,\ldots\]
        \textbf{Answer: }We apply the definition of a MGF:
            \begin{align*}
                M_X(t) &= E[e^{tw}] = \sum_{\text{all $k$}} e^{tk} p_W(k) = \sum_{k=0}^\infty e^{kt}\bigg(\frac{3}{4}\bigg)^k\bigg(\frac{1}{4}\bigg) = \frac{1}{4}\sum_{k=0}^\infty \bigg(\frac{3e^t}{4}\bigg)^k\\
                &= \frac{1}{4}\bigg(\frac{1}{1-\frac{3e^t}{4}}\bigg) \qquad \text{(By properties of a geometric series, where $r = \frac{3e^t}{4}$)}\\
                &= \frac{1}{4-3e^t}.
            \end{align*}
        
        \item 3.12.8. Let $Y$ be a continuous random variable with $f_Y(y) = ye^{-y}, \, 0 \leq y$. Show that $M_Y(t) = \frac{1}{(1-t)^2}$.\\
        \textbf{Answer: }
            \begin{align*}
                M_Y(t) &= \int_{-\infty}^\infty e^{ty}ye^{-y} \dif y = \int_0^\infty ye^{y(t-1)} \dif y
            \end{align*}
        Now we integrate by parts with $u=y,\,\dif u=1,\,v=\frac{\exp{y(t-1)}}{t-1},\, \dif v = e^{y(t-1)}$.
            \begin{align*}
                \int_0^\infty ye^{y(t-1)} \dif y &= \frac{ye^{y(t-1)}}{t-1}-\int_0^\infty \frac{e^{y(t-1)}}{t-1} \dif y = -\frac{1}{t-1}\int_0^\infty e^{y(t-1)} \dif y\\
                &= -\frac{1}{t-1}\bigg(\frac{e^{y(t-1)}}{t-1}\bigg)\bigg|_0^\infty = \frac{1}{(t-1)^2}.
            \end{align*}
        
        \item 3.12.12. What is $E(Y^4)$ if the random variable $Y$ has the moment-generating function $M_Y(t) = (1-\alpha t)^{-k}$?\\
        \textbf{Answer: }We must find the 4th derivative of the MGF and then evaluate this at $t=0$.
            \begin{align*}
                M_Y'(t) &= -k(1-\alpha t)^{-k-1}(-\alpha) = \alpha k(1-\alpha t)^{-k-1}\\
                M_Y''(t) &= (\alpha k)(-k-1)(1- \alpha t)^{-k-2}(-\alpha) = (\alpha^2k^2+\alpha^2k)(1- \alpha t)^{-k-2}\\
                M_Y^{(3)}(t) &=  (-k-2)(\alpha^2k^2+\alpha^2k)(1-\alpha t)^{-k-3}(-\alpha) = (\alpha^3k^3+3\alpha^3k^2+2\alpha^3k)(1-\alpha t)^{-k-3}\\
                M_Y^{(4)}(t) &= (-k-3)(\alpha^3k^3+3\alpha^3k^2+2\alpha^3k)(1-\alpha t)^{-k-4}(-\alpha) = (\alpha^4k^4+6\alpha^4k^3+11\alpha^4k^2+6\alpha^4k)(1-\alpha t)^{-k-4}\\
                M_Y^{(4)}(0) &= (\alpha^4k^4+6\alpha^4k^3+11\alpha^4k^2+6\alpha^4k)1^{-k-4} = E[Y^4].
            \end{align*}
        
        \item 3.12.20. Calculate $P(X \leq 2)$ if $M_X(t) = \big( \frac{1}{4}+\frac{3}{4}e^t\big)^5$.\\
        \textbf{Answer: }We first note that this MGF resembles that of a binomial random variable where $M_X(t) = (1-p-pe^t)^n$. In fact, if we let $n=5$ and $p=\nicefrac{3}{4}$, we can calculate this probability. We will calculate $P(X=0),\,P(X=1),$ and $P(X=2)$ separately.
            \begin{align*}
                \text{Note: }p_X(x) &= {n \choose x}p^x(1-p)^{n-x}\\
                P(X=0) &= \frac{5!}{0!(5-0)!} \bigg(\frac{3}{4}\bigg)^0 \cdot \bigg(1-\frac{3}{4}\bigg)^{5-0} \approx 0.001\\
                P(X=1) &= \frac{5!}{1!(5-1)!} \bigg(\frac{3}{4}\bigg)^1 \cdot \bigg(1-\frac{3}{4}\bigg)^{5-1} \approx 0.015\\
                P(X=2) &= \frac{5!}{2!(5-2)!} \approx  \bigg(\frac{3}{4}\bigg)^2 \cdot \bigg(1-\frac{3}{4}\bigg)^{5-2} \approx 0.088\\
            \therefore P(X \leq 2) &\approx 0.001+0.015+0.088 = 0.104.
            \end{align*}
    \end{enumerate}
\end{document}